\documentclass{scrartcl}
\usepackage{selinput}
\SelectInputMappings{
  adieresis={ä},
  germandbls={ß},
}
\usepackage[ngerman]{babel}
\usepackage[T1]{fontenc}
\usepackage{mathtools}
\usepackage{amsfonts}
\usepackage{amssymb}
\usepackage{enumitem}
\usepackage{stmaryrd}

\newcommand{\s}[1]{\overset{\sim}{#1}} % Funktion schlange
\newcommand{\q}[1]{\overline{#1}} % Funktion quer
\newcommand{\ub}[2]{\underbrace{#1}_{#2}} % \underbrace
\newcommand{\ob}[2]{\overbrace{#1}^{#2}} % \overbrace
\newcommand{\us}[2]{\underset{#1}{#2}} % \underset
\newcommand{\os}[2]{\overset{#1}{#2}} % \overset
\newcommand{\N}{\mathbb{N}}
\newcommand{\Z}{\mathbb{Z}}
\newcommand{\Q}{\mathbb{Q}}
\newcommand{\R}{\mathbb{R}}
\newcommand{\C}{\mathbb{C}}
\newcommand{\K}{\mathbb{K}}
\newcommand{\F}{\mathbb{F}}
\newcommand{\im}{\operatorname{im}}

\setcounter{MaxMatrixCols}{20}

\everymath{\displaystyle}

\begin{document}

\subject{Skriptum}
\title{Algebra I}
\subtitle{SS 2018}
\date{\today}
\maketitle

\tableofcontents

\section{Vektorräume}
\label{sec:vektorraeume}

\paragraph{Definition 1.1}
Sei $K$ ein Körper. Eine Menge $V$ heißt \textit{Vektorraum} über K wenn sie mit
folgenden Strukturen ausgestattet ist:
\begin{itemize}
\item $+ : V \times V \to V $ (Addition)
\item $\cdot~: K \times V \to V $ (Skalarenmultiplikation)
\end{itemize}
Mit folgenden Eigenschaften: \\
$\forall u,v,w \in V~\forall \lambda, \mu \in K :$
\begin{itemize}
\item $u + v = v + u$ (Kommutativität der Addition)
\item $(u + v) + w = u + (v + w)$ (Assoziativität der Addition)
\item $\exists 0 \in V: v + 0 = v$ (Neutrales Element der Addition)
\item $\exists v' \in V: v + v' = 0$ (Additives Inverses)
\item $\exists 1 \in K: 1 \cdot v = v$ (Neutrales Element der
  Skalarenmultiplikation) 
\item $(\lambda \cdot \mu) \cdot v = \lambda \cdot (\mu \cdot v)$
  (Assoziativität der Skalarenmultiplikation)
\item $\lambda \cdot (u + v) = \lambda \cdot u + \lambda \cdot v$
  (Distributivität I)
\item $v \cdot (\lambda + \mu) = v \cdot \lambda + v \cdot \mu$
  (Distributivität II)
\end{itemize}

\paragraph{Bemerkung 1.2}
Elemente von $V$ nennt man Vektoren. Elemente von $K$ nennnt man Skalare.
Außerdem ist das Nullelement eindeutig.
~\\\\
Im folgenden sei $K$ ein Körper und $V$ ein Vektorraum über $K$.
\paragraph{Definition 1.3}
$x_1, \dots x_n \in V$ heißen \textit{linear unabhängig}, wenn
$\exists \lambda_1, \dots, \lambda_n \in K$ und nicht alle gleich null sind,
sodass $\lambda_1x_1 + \dots + \lambda_nx_n = 0$.

\subsection{Basis und Dimension}
\label{subsec:basisunddimension}

\paragraph{Definition 1.4}
\begin{itemize}
\item Ein $V$ Vektorraum heißt \textit{endlichdimensional}, wenn $\exists N \in
\mathbb{Z}_{\geq0}$, sodass $\forall n > N$ und beliebige $v_1, \dots, v_n \in
V$ linear abhängig sind.
\item Kleinste solche $N$ nenne \textit{Dimension} von $V$.
\end{itemize}

\paragraph{Definition 1.5}
$v_1, \dots v_n \in V$ heißen \textit{Basis}, wenn
\begin{itemize}
\item linear unabhängig
\item $\forall u \in V \exists$ eindeutige $\lambda_1, \dots, \lambda_n \in K$,
  sodass $u = \lambda_1v_1 + \dots + \lambda_nv_n$
\end{itemize}

\paragraph{Lemma 1.6}
Sei $V$ ein $K$ Vektorraum mit $dimV = N$
\begin{enumerate}
\item $\exists$ Basis mit $N$ Elementen.
\item Jede Basis hat $N$ Elemente.
\end{enumerate}
\textit{Beweis} 1. klar 2. ohne Beweis. 

\subsection{Lineare Abbildungen}
\label{subsec:lineareabbildungen}

\paragraph{Definition 1.7}
Sei $V,W$ Vektorräume über $K$. $f: V \to W$ heißt \textit{Lineare Abbildung}
gdw.
\begin{itemize}
\item $\forall u,v \in V: f(v + u) = f(v) + f(u)$
\item $\forall \lambda \in K \forall v \in V: f(\lambda v) = \lambda f(v)$
\end{itemize}

\section{Matrizen}
\label{sec:matrizen}

\paragraph{Definition 2.1} Sei $K$ ein Körper. Eine $m \times n$ Matrix über $K$
(oder mit Einträgen in $K$) ist eine Tabelle:
\[
  A = \begin{pmatrix}
      a_{11} & \dots & a_{1n} \\
      \vdots & \ddots & \vdots \\
      a_{m1} & \dots & a_{mn}
    \end{pmatrix} = \left( a_{ij} \right) ~~ a_{ij} \in K
\]

\subsection{Operationen mit Matrizen}
\label{subsec:operationenmitmatrizen}

Sei im folgenden $A = (a_{ij}), B = (b_{ij}) \in Mat_{m,n}(K)$.
\begin{itemize}
\item $A + B = a_{ij} + b_{ij}$ (Addition)
\item $\lambda \in K: \lambda \cdot A = (\lambda \cdot a_{ij})$
  (Skalarenmultiplikation)
\item $C = (c_{ij}) \in Mat_{m,r}(K): A \cdot C = \left(
    \sum_{s=1}^{n}a_{is}c_{si} \right) \in Mat_{m,r}(K)$
  (Matrizenmultiplikation)
\item Transposition: $A^t := (a_{ji})$
\item Determinante: $det(A) = \sum_{i_1, \dots, i_n}(-1)^{inv(i_1, \dots,
    i_n)}a_{1_{i1}} \dots a_{n_{in}}$ mit $i_1, \dots, i_n$ Permutationen aus
  $\{1, \dots, n\}$
   
\item Inverse Matrix: $adj(A) = \left( (-1)^{i+j}M_{ij} \right)^t$
  \paragraph{Anmerkung:} Es gilt $A \cdot
  adj(A) = adj(A) \cdot A = det(A) \cdot Id_n$ und falls $det(A) \neq 0: A^{-1}
  = \frac{adj(A)}{det(A)}$
\end{itemize}

\subsection{Lineare Abbildungen und Matrizen}
\label{subsec:lineareabbildungenundmatrizen}

$\forall A \in Mat_{m,n}(K) \exists ! \psi_A: K^n \to K^m, x \mapsto Ax$. Es gibt
also einen Isomorphismus
\[Hom(K^n,K^m) \overset{\simeq}{\to} Mat_{m,n}(A)\]

\section{Monoide}
\label{sec:monoide}
\paragraph{Definition 3.1} Sei $M$ eine Menge. Ein Monoid ist ein Tripel $\left(
  M, \circ, e \right)$ mit: \\
$\circ: M \times M \to M, (x, y) \mapsto x \circ y$
Mit den folgenden Eigenschaften:
\begin{itemize}
\item $\forall x,y,z \in M: \left( x \circ y \right) \circ z = x \circ \left(
    y \circ z \right)$
\item $\forall x \in M: x \circ e = x$
\end{itemize}

\paragraph{Beispiele 3.2}
\begin{enumerate}
\item $ \left( \mathbb{Z}, +, 0 \right) $ ist ein Monoid.
\item $V$ Vektorraum über $K$: $\left( v, +, 0 \right)$ ist ein Monoid.
\item $\left( \mathbb{Z}, \cdot, 1 \right)$ ist ein Monoid.
\item $ \left( Mat_{n,n}(K), \cdot, Id_n \right)$ ist ein Monoid.
\end{enumerate}

\paragraph{Bemerkung 3.3}
Sei $M$ Monoid, dann ist das $1$ Element eindeutig.

\paragraph{Definition 3.4} $M$ heißt kommutativ $\leftrightarrow \forall x,y \in
M x \circ y = y \circ x$

\paragraph{Bemerkung 3.5} Wenn $M$ kommutativ ist dann schreibt man $x+y$
anstatt $x \circ y$. Außerdem (falls $x$ invertierbar):
\[
  \underset{n-mal}{x \circ \dots \circ x} = x^n ~,~ x^0 = e ~,~ \forall m,n \in
  \mathbb{Z}_{\geq 0}: x^n \circ x^m = x^{m+n}
\]

\paragraph{Definition 3.6} Sei $M$ Monoid. Dann heißt $M$ invertierbar gdw.
\[
  \exists y \in M: y \circ x = e ~\text{(links Inverse)}
\]
\[
  \exists z \in M: x \circ z = e ~\text{(rechts Inverse)}
\]

\paragraph{Lemma 3.7} Falls $x$ invertierbar ist, dann ist jedes links Inverse
gleich jedem rechts Inverse. \\
\textit{Beweis} $y = y \circ e = y \circ x \circ z = e \circ z = z \hfill
\square$

\paragraph{Bemerkung 3.8} $x$ invertierbar dann schreibe $x^{-1}$ (mult.) oder
$-x$ (additiv).

\section{Relationen}
\label{sec:relationen}

\paragraph{Definition 4.1} Sei $X$ eine Menge. Eine \textit{Relation} $R$ auf
$X$ ist eine Teilmenge $R \subset X \times X$. Man schreibt $x \sim_R y$, wenn
$(x,y) \in R$.

\paragraph{Beispiele 4.2}
\begin{enumerate}
\item $X =$ Menge aller Studierenden im Raum $R \subset X \times X: x \sim_R y
  \Leftrightarrow \text{$x$ mag $y$}$
\item $X = \mathbb{Z}~a \sim b \Leftrightarrow a < b$
\item $X = \mathbb{Z}~a \sim b \Leftrightarrow a - b \in 2\mathbb{Z}$
\end{enumerate}

\paragraph{Definition 4.3}
Sei $R$ Relation.
\begin{itemize}
\item $R~reflexiv~:\Leftrightarrow \forall x \in R: x \sim_R x$
\item $R~symmetrisch~:\Leftrightarrow \forall x,y \in R: x \sim_R y
  \Leftrightarrow y \sim_R x$
\item $R~transitiv~:\Leftrightarrow \forall x,y,z \in R: x \sim_R y \wedge y
  \sim_R z \Rightarrow x \sim_R z$
\end{itemize}

\subsection{Äquivalenzrelationen}
\label{subsec:aequivalenzeraltionen}

\paragraph{Definition 4.4}
Eine Relation $R$ heißt Äquivalenzrelation, gdw. $R$ reflexiv, transitiv und
symmetrisch ist.

\paragraph{Definition 4.5}
Sei $R$ Äquivalenzrelation auf $X$. $\forall x \in X: [x]_R = \{a \in X ~|~ a
\sim_R x\} \subset X$.
Nenne $[x]_R$ Äquivalenzklasse von $x$ bezüglich $R$. Mit Eigenschaften:
\begin{enumerate}
\item $\forall x,y \in R: [x]_R = [y]_R \Leftrightarrow y \in [x]_R$
\item $X$ zerfällt in disjunkte Vereinigung der Äquivalenzklassen.
\end{enumerate}
\paragraph{Beweis}
\begin{enumerate}{}
\item Ist nichts zu zeigen.
\item Da $R$ Reflexiv gilt für beliebiges $x \in X \Rightarrow x \in [x]_R$.
  Damit gilt bereits $X = \bigcup_{x \in X}[x]_{R}$. Es ist noch zu zeigen dass
  für beliebige $x_1,x_2 \in X: [x_1]=[x_2]$ oder $[x_1]\cap [x_2] = \emptyset$.
  Sei dazu $y \in [x_1]\cap [x_2]$. Somit gilt $yRx_1$ und $yRx_2$ und da $R$
  reflexiv und symmetrisch ist gilt auch $x_1Rx_2$. Damit muss aber auch $x_1
  \in [x_2]$ und $x_2 \in [x_1]$ sein. Damit ist $[x_1]=[x_2]$. Daraus folgt die
  Behauptung.
  \hfill $\square$
\end{enumerate}

\paragraph{Definition 4.6} Die Menge der Äquivalenzklassen wird mit $X/R$ oder
$X/\sim$ bezeichnet. Man nennt $X/\sim$ Quotient bzgl. $\sim$.

\paragraph{Beispiel 4.7} $a \sim b \Leftrightarrow a - b \in \mathbb{Z}$ ~~
$\mathbb{Z} = \{[0], [1], \dots, [n - 1]\}$

\section{Gruppen}
\label{sec:gruppen}

\paragraph{Definition 5.1} Eine Gruppe ist ein Monoid in dem jedes Element
invertierbar ist.

\paragraph{Lemma 5.2} Sei $M$ ein Monoid.
\[
  M^* = \{ x \in M ~|~ x ~\text{invertierbar} \} \subset M \Rightarrow M^*
  ~\text{ist Gruppe}
\]
\paragraph{Beweis}
Nach Definition ist eine Gruppe ein Monoid in dem jedes Element invertierbar
ist. Für beliebiges $m \in M^*: m^{-1} \in M$ laut Definition von $M^*$. Es
bleibt noch zu zeigen dass für $m_1,m_2\in M$ auch $m_1 \circ m_2 \in M$. Da
$m_1^{-1}, m_2^{-1} \in M \Rightarrow m_2^{-1} \circ m_1^{-1} = (m_1 \circ
m_2)^{-1}$ und somit
\[
  (m_1 \circ m_2)(m_1 \circ m_2)^{-1} = e.
\]
Daraus folgt die Behauptung.
\hfill $\square$
\paragraph{Beispiele 5.3}
\begin{enumerate}
\item $G = \{e\}$ ist Gruppe.
\item $\left( \mathbb{Z}, + \right)$ oder auch mit $\mathbb{Q, R, C}$ ist Gruppe.
\item $\left( \mathbb{Q}, \cdot \right)$ oder auch mit $\mathbb{R^\star,C^\star}$
\item $S' = \{ z \in \mathbb{C} ~|~ |z| = 1\} \left( z = e^{i\phi} ~~ z^{-1} =
    e^{-i\phi} \right)$
\item $M = \{ z \in \mathbb{C} ~|~ z^n = 1\} \subset S'$ $M$ hat $n$ Elemente.
\item $GL_n(K) = \{ A \in Mat_{n,n}(k) ~|~ det(A) \neq 0\}$
  \item $SL_n(K) = \{ A \in Mat_{n,n}(K) ~|~ det(A) = 1\}$
\end{enumerate}

\paragraph{Lemma 5.4} Sei $G$ eine Gruppe und $x,y,z \in G$, dann gilt:
\begin{enumerate}
\item $x \circ y = x \circ z \Rightarrow y = z$
\item $y \circ x = z \circ x \Rightarrow y = z$
\end{enumerate}
\paragraph{Beweis}
\begin{enumerate}
\item Gelte $x \circ y = x \circ z$ dann folgt $x^{-1} \circ x \circ z = x^{-1}
  \circ x \circ y \Rightarrow y = z$
\item analog
\end{enumerate}
\hfill $\square$

\paragraph{Definition 5.5} Sei $G$ eine Gruppe, dann heißt $G$ kommutativ oder
abelsch $\Leftrightarrow \forall x,y \in G: x \circ y = y \circ x$

\paragraph{Definition 5.6} Sei $G$ eine Gruppe, $H \subset G$ so nenne $H$
Untergruppe gdw.
\begin{itemize}
\item $e \in H$
\item $x,y \in H: x \circ y \in H$
\end{itemize}

\paragraph{Beispiele 5.7}
\begin{enumerate}
\item $n\mathbb{Z} \subset \mathbb{Z}$ mit Addition ist Untergruppe.
\item $\mathbb{Z} \subset \mathbb{Q} \subset \mathbb{R} \subset \mathbb{C}$
\item $\mathbb{Z}^\star \subset \mathbb{Q}^\star \subset \mathbb{R}^\star
  \subset \mathbb{C}^\star$
\item $GL_n(\mathbb{Z}) \subset GL_n(\mathbb{Q}) \subset \dots$
\item $SL_n(\mathbb{Z}) \subset \dots$
\end{enumerate}

\subsection{Gruppenhomomorphismen}
\label{subsec:gruppenhomomorphismen}

\paragraph{Definition und Lemma 5.8} Seien $G, H$ Gruppen und $f: G \to H$. Dann ist $f$
ein Gruppenhomomorphismus gdw.
\begin{enumerate}
\item $\forall a,b \in G: f(a \circ b) = f(a) \circ f(b)$
\item $f(e_G) = e_H$
\end{enumerate}
\paragraph{Beweis}
Zu 2.:
\[
  f(e_G)f(e_G) = f(e_Ge_G) = f(e_G)
  \Rightarrow \exists h \in H: h \cdot h = h
  \Leftrightarrow \ub{h^{-1} \cdot h}{= e_H} \cdot h = \ub{h^{-1} \cdot h}{=e_H}
  \Leftrightarrow h = e_H
\]

\paragraph{Definition 5.9}
\begin{enumerate}
\item Monomorphismus = injektiver Homomorphismus
\item Epimorphismus = surjektiver Homomorphismus
\item Isomorphismus = bijektiver Homomorphismus
\item Endomorphismus = $Hom(G, G) = End(G)$
\item Automorphismus = bijektiver Endomorphismus = $Aut(G)$
\end{enumerate}

\paragraph{Lemma 5.9.1}
Sei $G$ eine Gruppe und $f:G \to H$ ein Gruppenhomomorphismus dann gilt
\[
  \forall g \in G: f(g^{-1}) = f(g)^{-1}
\]
\paragraph{Beweis}
Betrachte dazu
\[
  f(g)f(g^{-1}) = f(gg^{-1}) = f(e_G) = e_H 
\]
$f(g^{-1})$ besitzt also genau die Eigenschaft des Inversen und da in einer
Gruppe das Inverse eindeutig bestimmt ist
($g^{-1}=g^{-1}e_G=g^{-1}gg'^{-1}=g'^{-1}$), folgt die Behauptung.
\hfill $\square$

\paragraph{Definition und Lemma 5.10}
Sei $f: G \to H \in Hom(G,H)$

\begin{align}
  &\ker(f) := \{x \in G ~|~ f(x) = e\} \subset G \\
  &\im(f) := \{y \in H ~|~ \exists x \in G: f(x) = y\} \subset H
\end{align}
Wobei $\ker(f)$ und $\im(f)$ Untergruppen sind.
\paragraph{Beweis}
\begin{enumerate}{}
  \item Wegen $f(e_G)=e_H$ ist $e_G \in \ker(f)$. Seien $g,g' \in \ker(f)$ dann
    gilt
    \[
      f(gg') = f(g)f(g') = e_He_H = e_H
    \]
    Es bleibt noch zu zeigen dass die Inversen in $\ker(f)$ enthalten sind. Für
    $g^{-1} \in \ker(f)$ ist
    \[
      f(g^{-1}) = f(g)^{-1} = e_H^{-1} = e_H
    \]
  \item Es ist $e_H=f(e_G) \in \im(f)$. Seien nun $g,g' \in G$ dann gilt
    \[
      f(g)f(g') = f(gg') \in H.
    \]
    Ebenso ist
    \[
      f(g)^{-1} = f(g^{-1}) \in H.
    \]
    \hfill $\square$
\end{enumerate}

\paragraph{Beispiel 5.11}
~\\ $\det: GL_n(K) \to K^*$ mit $\ker(\det) = \{A \in GL_n(K) ~|~ \det(A) = 1\} =
SL_n(K)$

\paragraph{Definition 5.12} Seien $G_1, G_2$ Gruppen und $A = G_1 \times G_2 =
\{(g_1,g_2) \in G_1 \times G_2\}$
\[
  A \times A \to A: ((g_1, g_2), (h_1, h_2)) \mapsto (g_1 \circ h_1, g_2 \circ
  h_2) ~\text{wobei}~ g_1 \circ h_1 \in G_1, g_2 \circ h_2 \in G_2
\]
definiert eine neue Gruppe. \\
\paragraph{Beweis}
\begin{itemize}
\item Assoziativität folgt aus Assoziativität von $G_1$ und $G_2$
\item $(e_{G1}, e_{G2}) = e_A$
\item $(g_1, g_2) = (g_1^{-1}, g_2^{-1})$
\end{itemize}

\paragraph{Beispiel 5.13}  $\mathbb{Z} \times \mathbb{Z} = \{ (a, b) ~|~ a,b \in
\mathbb{Z}\}$

\subsection{Gruppenwirkungen}
\label{subsec:gruppenwirkungen}

\paragraph{Definition 5.14} Sei $G$ eine Gruppe und $X$ eine Menge. Eine
\textit{Gruppenwirkung} von $G$ auf $X$ ist eine Abbildung:
\[
  \sigma: G \times X \to X, (g, x) \mapsto \sigma(g, x) = g \cdot x
\]
mit folgenden Eigenschaften:
\begin{itemize}
\item $\forall x \in X: \sigma(e, x) = x$
\item $\forall g_1,g_2 \in G~\forall x \in X: \sigma(g_1 \cdot g_2, x) =
  \sigma(g_1, \sigma(g_2, x))$
\end{itemize}

\paragraph{Beispiele 5.15}
\begin{enumerate}
\item $X = \{1, 2, \dots, n\} ~~ G = S_n $ Permutationsgruppe.
\item $G \times G \to G$
\item $
  \begin{pmatrix}
    a_{11} & \dots & a_{1n} \\
    \vdots & \ddots & \vdots \\
    a_{m1} & \dots & a_{mn}
  \end{pmatrix} \cdot
  \begin{pmatrix}
    x_1 \\
    \vdots \\
    x_n
  \end{pmatrix} \in G = GL_n(K)$
\end{enumerate}

\paragraph{Lemma 5.16} Sei $X$ eine Menge. Eine Gruppenwirkung definiert man
äquivalent als Gruppenhomomorphismus.
\[
  \rho: G \to Aut(X),~ g \mapsto \rho(g)
\]
\paragraph{Beweis} Steh aus.

\paragraph{Definition 5.17} Sei $G$ Gruppe und $X$ Menge mit $G \circlearrowleft
X$ dabei bezeichnet $\circlearrowleft$ eine Gruppenwirkung.
\begin{itemize}
\item Stabilisator: $x \in X: Stab(x) = \{g \in G ~|~ g \cdot x = x\}
  \subset G$
\item Orbit (Bahn): $x \in X: Orb(x) = \{y \in X ~|~ \exists g \in G: g \cdot x
  = y\}$
\item Eine Wirkung ist frei $:\Leftrightarrow \forall x \in X: Stab(x) = \{e\}$
\item Eine Wirkung heißt transitiv $:\Leftrightarrow \forall x,y \in X \exists g
  \in G: g \cdot x = y$
\end{itemize}

\paragraph{Bemerkung 5.18} Es gibt nur eine Bahn $\Leftrightarrow$ transitiv.
\paragraph{Bemerkung 5.19} $G \circlearrowleft X$ definiert eine
Äquivalenzrelation. Die Bahnen der Wirkung entsprechen den Äquivalenzklassen.
\paragraph{Bemerkung 5.20} Genau genommen ist $\circlearrowleft$ linke Wirkung.
Es gibt auch rechte Wirkungen (Bsp.: Matrizenmultiplikation von links.).

\subsection{Nebenklassen}
\label{subsec:nebenklassen}

\paragraph{Definition 5.21} Sei $H \subset G$ Untergruppe. Die Relation
\[
  x \sim y \Leftrightarrow \exists h \in H: x \cdot h = y
\]
definiert eine Äquivalenzrelation. Die Äquivalenzklassen dieser Relation
\[
  xH = \{xh ~|~ h \in H\} \subset G
\]
nennt man \textit{Linksnebenklassen}. Die Gruppe $G$ zerfällt in disjunkte
Vereinigung von $H$ Linksnebenklassen.
Bezeichne die Menge der $H$ Linksnebenklassen mit $G/H$ ($G$ modulo $H$).
Analog definiert man $H$ Rechtsnebenklassen.

\paragraph{Beispiele 5.22}
\begin{enumerate}
\item Jede Untergruppe von $\mathbb{Z}$ ist in der Form $n\mathbb{Z}, n \geq 0$
\item $\mathbb{Z}/n\mathbb{Z} = \{0 + n\mathbb{Z}, \dots, (n - 1) + n\mathbb{Z}\}$
\end{enumerate}

\paragraph{Satz 5.23} (\textit{Satz von Langrange}) Sei $G$ endl. Gruppe und $H$
Untergruppe von $G$ dann gilt:
\[
  |G| = |H| \cdot |G/H|
\]
\paragraph{Beweis}
Die Anzahl der Rechtsnebenklassen entspricht der Anzahl der Linksnebenklassen.
Die Gruppe $G$ zerfällt in eine disjunkte Vereinigung von Nebenklassen
$n=|G/H|$, wobei es zwischen zwei Nebenklassen immer eine Bijektion gibt. Somit
sind alle Nebenklassen gleichmächtig, mit Mächtigkeit $r=|H|$. Damit gilt
\[
  r \cdot n = |G|,
\]
und damit die Behauptung.
\hfill $\square$

\paragraph{Lemma 5.24} \textit{Betrachten wir eine transitive Wirkung von $G$
  auf $X$. Für jedes $P \in X$ können wir eine sujektive Abbildung definieren}
\[
  f_P: G \to X,~~g \mapsto g \cdot P,
\]
\textit{die eine Bijektion}
\[
  \phi: G/Stab(P) \to X,~~gStab(P) \mapsto g \cdot P
\]
\textit{induziert.}
\paragraph{Beweis}
Da die Wirkung transitiv ist, muss die Abbildung $f_P$ surjektiv sein und somit
$\phi$. Für die Injektivität betrachte folgende Gleichung mit $g_1,g_2 \in G$
\[
  g_1 \cdot P = g_2 \cdot P = P \iff g_2^{-1} \cdot g_1 \cdot P = P \implies
  g_2^{-1}g_1 \in Stab(P).
\]
Damit werden in $\phi$ alle Elemente in $g_1, g_2 \in G$ mit $g_1 \neq g_2$ für
die galt $f_P(g_1)=f_P(g_2)=P$ zusammengefasst zu einem Element. Damit ist die
Abbildung $\phi$ injektiv.
\hfill $\square$

\subsection{Normalteiler}
\label{subsec:normalteiler}

\paragraph{Lemma 5.25} Sei $H \subset G$ Untergruppe. Die folgenden Aussagen
sind äquivalent:
\begin{enumerate}
\item $\forall g \in G: gHg^{-1} = H$
\item $\forall g \in G \forall h \in H: ghg^{-1} \in H$
\item $\forall g \in G: gH = Hg$
\item $H$ Linksnebenklassen = $H$ Rechtsnebenklassen
\end{enumerate}
\paragraph{Beweis}
\begin{itemize}{}
\item 3. $\iff$ 4. Ist klar.
\item 1. $\iff$ 2. Ebenfalls klar.
\item 1. $\iff$ 4. Betrachte die Gleichung
  \[
    gH = Hg \iff g^{-1}gH=g^{-1}Hg \iff H = g^{-1}Hg
  \]
\end{itemize}
\hfill $\square$

\paragraph{Definition 5.26} Eine Untergruppe $H \subset G$ heißt \textit{normal}
oder \textit{Normalteiler} gdw. eine der Aussagen aus Lemma 5.25 gilt.

\paragraph{Beispiele 5.27}
\begin{enumerate}
\item Jede Untergruppe in abelscher Gruppe ist normal.
\item $SL_n(K) \subset GL_n(K)$ ist normal.
\item $
  \begin{pmatrix}
    * & * \\
    0 & *
  \end{pmatrix} \subset GL_2(K)
  $ ist nicht normal.
\end{enumerate}

\paragraph{Lemma 5.28} Sei $\psi: G \to H$ ein Gruppenhomomorphismus dann folgt
$\ker(\psi)$ ist Normalteiler.
\paragraph{Beweis}
Wissen bereits das $\ker(\psi)$ Untergruppe ist. Betrachte dann die Gleichung
mit $g \in G, h \in \ker(\psi)$
\[
  \psi(ghg^{-1})=\psi(g)\ub{\psi(h)}{=e_H}\psi(g^{-1})
  =\psi(g)\psi(g^{-1})=\psi(gg^{-1})=\psi(e_G)=e_H \in \ker(\psi)
\]
\hfill $\square$

\subsection{Faktorgruppen}
\label{subsec:faktorgruppen}

\paragraph{Definition 5.29} Ist $N \subset G$ Normalteiler, so macht die
Verknüpfung
\[
  G/N \times G/N \to G/N,~(gN, hN) \mapsto (gh)N
\]
$G/N$ zu einer Gruppe mit neutralem $eN$ und Inversem $g^{-1}N$ zu $gN \in G/N$.
$G/N$ wird Faktorgruppe oder Quotient von $G$ bzgl. $N$ genannt.
\paragraph{Bemerkung 5.28} Die Assoziativität der Verknüpfung ist klar. Auserdem
ist die Verknüpfung wohldefiniert.
\paragraph{Beweis}
Da $N$ Normalteiler ist, gilt $\forall g \in G: gN=Ng$, also insbesondere gibt
es für jedes $n \in N$ ein $\s{n} \in N$ mit $gn=\s{n}g$. Genauer ist $\s{n}=gng^{-1}$. 
Sei nun $g_1'N=g_1N$ wobei $g_1'=g_1n_1$ und $g_2'N=g_2N$ wobei $g_2'=g_2n_2$
mit $n_1n_2 \in N$ dann betrachte die Gleichung
\[
  (g_1'N)(g_2'N) = (g_1'g_2'N) = (g_1n_1g_2n_2)N = (g_1g_2\s{n_1}n_2)N
  =(g_1g_2)N = (g_1N)(g_2N)
\]

\subsection{Einige wichtige Sätze der elementaren Gruppentheorie}
\label{subsec:einigewichtigesaetzederelementaregruppentheorie}

\paragraph{Satz 5.29} (\textit{Homomorphiesatz}) Sei $f: G \to H$ ein
Gruppenhomomorphismus. Dann ist
\[
  \overline{f}: G/ker(f) \to H,~\overline{f}([g]) = f(g)
\]
ein wohldefinierter injektiver Gruppenhomomorphismus. \\
\paragraph{Beweis}
Zunächst gilt $[g]=[\s{g}] \iff \exists h \in \ker(f): g = h\s{g}$.
Betrachte damit
\[
  \q{f}([g])=f(g)=f(h\s{g})=f(h)f(\s{g})=f(\s{g})=\q{f}([\s{g}])
\]
Damit ist die Abbildung wohldefininert. Die Homomorphie Eigenschaft ist auch
klar. $\q{f}$ ist genau dann injektiv wenn
gilt $\ker(\q{f})=\{[e_G]\}$. Es gilt
\[
  \ker(\q{f})=\{[g] \in G/\ker(f)~|~\q{f}(g)=f(g)=e_H\}
  =\{[g] \in G/\ker(f)~|~g \in \ker(f)\}
  =\{[e_H]\}.
\]
Also ist $\q{f}$ injektiv.
\hfill $\square$

\paragraph{Bemerkung 5.30}
\begin{itemize}
\item Sei $f: G \to H$ Gruppenhomomorphismus und $N \subset
  G$ ein Normalteiler mit $N \subset ker(f)$, dann ist $\overline{f}: G/N \to H,
  \overline{f}([g]) = f(g)$ ein wohldefinierter Gruppenhomomorphismus.
  \paragraph{Beweis}
  Klar nach Beweis von Homomorphiesatz.
  \hfill $\square$

\item Sei $f: G \to H$ Gruppenhomomorphismus und $\overline{f}$ wie oben. \\
  Dann ist $\q{f}:G/\ker(f) \to \im(\overline{f})$ ein Isomorphismus.
  \paragraph{Beweis}
  Es gilt nach Definition von $\q{f}$, $\im(\q{f})=\im(f)$. Da eine Abbildung
  immer surjektiv zu ihrem Bild ist folgt die Behauptung.
  \hfill $\square$
  
\end{itemize}

\paragraph{Beispiel 5.31} Betrachte den Gruppenhomomorphismus $f: \mathbb{R} \to
\mathbb{C}\backslash\{0\},~t \mapsto e^{2\pi it}$. Dann ist $ker(f) =
\mathbb{Z}$ und $\im(f) = S^1$. Nach Homomorphiesatz ist also:
\[
  f: \mathbb{R}/\mathbb{Z} \to S^1,~[t] \mapsto e^{2\pi it}
\]
ein Gruppenisomorphismus.

\paragraph{Satz 5.32} (\textit{Korrespondenzsatz}) Seien $G$ eine Gruppe und $N
\subset G$ ein Normalteiler. Setze
\[
  \mathcal{U} := \{U \subset G ~|~ N \subset U\},
\]
die Menge der Untergruppen die $N$ enthalten. Weiter sei
\[
  \mathcal{V} := \{V \subset G/N ~\text{Untergruppe}\},
\]
die Menge der Untergruppen von $G/N$.
Dann ist
\[
  a: \mathcal{U} \to \mathcal{V},~U \mapsto \pi(U)
\]
und
\[
  b: \mathcal{V} \to \mathcal{U},~V \mapsto \pi^{-1}(V)
\]
eine Bijektion, wobei
\[
  \pi: G \to G/N,~g \mapsto [g]
\]
Es gilt zudem $U \in \mathcal{U}$ ist Normalteiler in $G \Leftrightarrow a(U)
\subset G/N$ ist Normalteiler.
\paragraph{Beweis}
$\pi$ ist ein Gruppenhomomorphismus somit ist $\pi(U) \subset G/N$ eine
Untergruppe für alle $U \in \mathcal{U}$, somit ist $a$ wohldefiniert. Zur
Wohldefiniertheit von $b$ ist zu zeigen dass $\pi^{-1}(V) \subset G$ auch für
alle $V \in \mathcal{V}$ eine Untergruppe ist. \\
Es ist $e_GN \in V$ f.a. $V \in \mathcal{V}$. Damit gilt
\[
  N = \pi^{-1}(e_GN) \subset \pi^{-1}(V).
\]
Seien nun $g,h \in \pi^{-1}(V)$ beliebig. Es ist zu zeigen dass $gh \in
\pi^{-1}(V)$. Dazu
\[
  \pi(g)\pi(h)=\pi(gh).
\]
Damit ist $gh \in \pi^{-1}(V)$. Weiter ist
\[
  \pi(e_G)=e_GN.
\]
Also auch $e_G \in \pi^{-1}(V)$. Die Inversen sind auch dabei, denn es gilt für
beliebges $g \in \pi^{-1}(V)$
\[
  \pi(g^{-1})=\pi(g)^{-1}.
\]
Somit ist $N \subset \pi^{-1}(V)$ eine Untergruppe mit Normalteiler $N$ wie gefordert.
Es bleibt noch zu zeigen dass $a \circ b = id_{\mathcal{V}}$ und $b \circ a =
id_{\mathcal{U}}$. Für dass erste ist zu zeigen
\[
  a \circ b(V) = \pi(\pi^{-1}(V)) = V.
\]
Dabei ist klar dass $\pi(\pi^{-1}(V)) \subset V$ gilt, denn das gilt für jede
Abbildung. Da $\pi$ surjektiv ist, gilt auch $\pi(\pi^{-1}(V)) \supset V$. Somit
$\pi(\pi^{-1}(V)) = V$. Bleibt noch zu zeigen
\[
  b \circ a(U) = \pi^{-1}(\pi(U)) = U.
\]
Wegen Surjektivität von $\pi$ ist $\glqq \supset \grqq$ klar. Für $\glqq \subset
\grqq$: Es gilt
\begin{align*}
  g \in \pi^{-1}(\pi(U)) &\iff gN = \pi(g) \in \pi(U) \\
                         &\iff \exists u \in U: gN = uN = \pi(u) \\
                         &\iff \exists n \in N: g = un
\end{align*}
Es bleibt noch zu zeigen: $U \in \mathcal{U}$ ist Normalteiler in $G \Leftrightarrow a(U)
\subset G/N$ ist Normalteiler. Es gilt für $U \in \mathcal{U}$
\begin{align*}
  U \subset G~\text{Normalteiler} &\iff \forall g \in G: gUg^{-1} = U \\
                                  &\implies \forall gN = \pi(g) \in G/N: (gN)\pi(U)(gN)^{-1}
                                    = [g]\pi(U)[g]^{-1} =\pi(gUg^{-1}) = \pi(U).
\end{align*}
Umgekehrt sei $V \in \mathcal{V}$
\begin{align*}
  V \subset G/N~\text{Normalteiler} &\iff gN \in G/N: (gN)V(gN)^{-1} = V \\
                                    &\implies \forall g \in G, \forall u \in \pi^{-1}(V):
                                      \pi(gug^{-1}) = \pi(g)\pi(u)\pi(g)^{-1} \in V
\end{align*}
Was genau die Normalteilereigenschaft von $\pi^{-1}(V)$ zeigt.
\hfill $\square$

\paragraph{Notation 5.33} Ist $G$ eine Gruppe, $N \in G$ ein Normalteiler und $H
\subset G$ Untergruppe, dann setze $N \cdot H = \{n \in H, h \in H\}$. $N \cdot
H$ ist eine wohldefinierte Untergruppe.

\paragraph{Satz 5.34} (\textit{Isomorphiesatz}) Sei $G$ eine Gruppe.
\begin{enumerate}
\item Ist $H \subset G$ eine Untergruppe und $N \subset G$ ein Normalteiler.
  Dann ist $N \subset N \cdot H$ und $H \cap N \subset H$ Normalteiler und es
  existiert ein kanonischer Isomorphismus
  \[
    H/(N \cap H) \overset{\simeq}{\to} (NH)/N
  \]
  .
\item Sind $N, H \subset G$ Normalteiler mit $H \subset N$ insbesondere ist also
  auch $H \subset N$ Normalteiler. Dann ist auch $N/H \subset G/H$ Normalteiler
  und es existiert ein kanonischer Isomorphismus
  \[
    G/N \overset{\simeq}{\to}(G/H)/(N/H)
  \]
  .
\end{enumerate}
\paragraph{Beweis}
\begin{enumerate}{}
\item Dass $N \subset N \cdot H$ Normalteiler ist, ist klar. Betrachte den
  Gruppenhomomorphismus
  \[
    f: H \longrightarrow N \cdot H \longrightarrow N \cdot H/N,
    ~h \longmapsto e_Gh \longmapsto (e_Gh)N. 
  \]
  Dieser ist surjektiv. Denn für jedes $(nh)N$ gilt
  \[
    (nh)N = (ne_G)N \cdot (he_G)N = (he_G)N = f(h).
  \]
  Nach dem Homomorphiesatz gilt wegen
  \[
    \ker(f) = \{h \in H ~|~ hN = (e_Gh)N = e_GN \in N \cdot H/N\} = \{h \in H |
    h \in N\} = N \cap H
  \]
  die Behauptung.

\item $N/H \subset G/H$ Normalteiler ist klar nach Korrespondenzsatz, da gilt $N
  \subset G$ Normalteiler. Betrachte den Gruppenhomomorphismus
  \[
    f: G \longrightarrow G/H \longrightarrow (G/H)/(N/H),~
    g \longmapsto gH \longmapsto [gH].
  \]
  Dieser ist offensichtlich surjektiv. Nach dem Homomorphiesatz genügt
  \[
    \ker(f) = \{g \in G ~|~ [gH] = [e_GH]\} = \{g \in G ~|~ gH \in N/H\} = N.
  \]
  Somit ist die Behauptung erfüllt.
\end{enumerate}
\hfill $\square$

\subsection{Zyklische Gruppen}
\label{subsec:zyklischegruppen}

Sei im folgenden $G$ eine Gruppe und $M \subset G$ eine Teilmenge.

\paragraph{Definition 5.35}
Bezeichne mit $\langle M \rangle$ die kleinste Untergruppe von $G$ die $M$
enthält.

\paragraph{Bemerkung 5.36}
Es gibt in der Tat eine eindeutige solche Untergruppe, denn: Sind $U,V \in G$
Untergruppen mit $M \subset U, M \subset V$,  so gilt $M \subset U \cap V$.

\paragraph{Bemerkung 5.37}
Die $\langle M \rangle$ besteht aus allen Elementen der Form
\[
  m_1^{i_1}m_2^{i_2} \dots m_n^{i_n} ,~~ n \in \mathbb{Z}_{\geq
    0},~~
  \begin{rcases}
    m_j \in M \\
    i,j \in \{\pm 1\}
  \end{rcases} = j = 1, \dots, n
\]
Wichtig: $m_i = m_j$ für $i \neq j$ ist natürlich erlaubt.

\paragraph{Notation 5.38}
Ist $M = \{m_1, \dots, m_n\}$ für ein $n \in \mathbb{Z}_{\geq 0}$ schreibe auch
$\langle M \rangle = \langle m_1, \dots, m_n \rangle$

\paragraph{Definition 5.39}
Eine Gruppe $G$ heißt \textit{zyklisch} gdw. es ein $g \in G$ gibt, sodass
\[
  G = \langle {g} \rangle = \{\dots, g^{-1}, e, g^1, \dots\}
\]
.

\paragraph{Definition 5.40}
Für ein Element $g \in G$ heißt $ord(g) := |\langle {g} \rangle|$ die Ordnung
von G.

\paragraph{Beispiele 5.41}
\begin{enumerate}
\item $\mathbb{Z} = \langle {1} \rangle$
\item $\mathbb{Z}/n\mathbb{Z} = \langle {[1]} \rangle$ für $n \in \mathbb{Z}_{>
    1}$
\item Mit dem Satz von Langrange folgt: Ist $|G| = p$ für eine Gruppe und
  Primzahl $p$, so ist $G$ zyklisch.
\end{enumerate}

\paragraph{Lemma 5.42}
Sei $G = \langle {g} \rangle$ zyklische Gruppe
\begin{enumerate}
\item Jede Untergruppe von $G$ ist wieder zyklisch.
\item Ist $|G| = n$, so existiert für jedes $m|n$ genau eine Untergruppe \\ $U_m
  \subset G$ mit $|U_m| = m$.
\end{enumerate}
\textit{Beweis} $\dots$

\paragraph{Beispiel 5.43}
Sei $G = \langle {g} \rangle$ zyklische Gruppe Dann ist
\[
  f: \mathbb{Z} \to G,~i \mapsto g^i
\]
ein surjektiver Gruppenhomomorphismus. Als Untergruppe der zyklischen Gruppe
$\mathbb{Z}$, ist also $ker(f)$ zyklisch, $ker(f) = \langle {n} \rangle =
n\mathbb{Z}$ für ein $n \in \mathbb{Z}$.
Nach Homomorphiesatz haben wir also einen Isomorphismus
\[
  \overline{f}:\mathbb{Z}/n\mathbb{Z} \overset{\simeq}{\to} G
\]
.
\subsection{Symmetrische Gruppen}
\label{subsec:symmetrischegruppen}

\paragraph{Notation 5.44}
Setze für $n \in \mathbb{Z}_{\geq 0}$~~$\triangle_n := \{1,2, \dots, n\}$

\paragraph{Definition 5.45}
Die Symmetrische Gruppe $S_n$ sei definiert als
\[
  S_n := Aut(\triangle_n) = (\{f: \triangle_n \to \triangle_n ~bijektiv\}, 0)
\]
.

\paragraph{Bemerkung 5.46}
$|S_n| = n!$

\paragraph{Notation 5.47}
Stelle $\sigma \in S_n$ dar durch
\[
  \begin{pmatrix}
    1         & 2         & 3         & \dots & n         \\
    \sigma(1) & \sigma(2) & \sigma(3) & \dots & \sigma(n)
  \end{pmatrix}
\]
\begin{itemize}
\item Ein Element $\tau \in S_n$ heißt \textit{k-Zykel} gdw. es $\{x_1, \dots,
  x_k\} \in \triangle_n$ gibt sodass
  \[
    \begin{matrix}
      &\tau(x_i) = x_{i+1} ~\text{für}~ i < k \\
      &\tau(x_k) = x_1
    \end{matrix}
  \]
  Wir notieren einen solchen Zykel als $(x_1, x_2, \dots, x_k)$.
  \item Zwei Zykel $(x_1, \dots, x_k)$ und $(y_1, \dots, y_s)$ heißen disjunkt,
    gdw. die Menge $\{x_1, \dots, x_k\}$ und $\{y_1, \dots, y_s\}$ disjunkt sind.
\end{itemize}
.

\paragraph{Bemerkung 5.48}
Sind $\tau$ und $\sigma$ disjunkte Zykel, so gilt $\tau \circ \sigma = \sigma
\circ \tau$. Ein Zykel der Länge zwei wird \textit{Transposition} genannt.

\subsection{Eigenschaften symmetrischer Gruppen}
\label{subsec:eigenschaftensymmetrischergruppen}

\paragraph{Satz 5.49}
\begin{enumerate}
\item Jedes Element $\sigma \in S_n$ lässt sich eindeutig (bis auf die
  Reihenfolge der Faktoren) als Produkt disjunkter Zykel schreiben.
\item Jedes $\sigma \in S_n$ ist ein Produkt von Transpositionen. D.h. $S_n =
  \langle {T} \rangle$ für $T$ die Menge der Transpositionen in $S_n$.
\end{enumerate}
\textit{Beweis} $\dots$

\paragraph{Beispiel 5.50}
\[
  \sigma =
  \begin{pmatrix*}[r]
    1 & 2 & 3 & 4 & 5 & 6 & 7 & 8 & 9 & 10 & 11 \\
    3 & 8 & 5 & 4 & 6 & 1 & 11 & 9 & 2 & 10 & 7
  \end{pmatrix*}
  = (1, 3, 5) \circ (2, 8, 9) \circ (7, 11)
\]

\paragraph{Korollar 5.51}
Für $ n \geq 2$ ist:
\[
  S_n = \langle {(1, 2), (1, \dots, n)} \rangle = \langle {(i, i+1), (1, \dots,
    n)} \rangle
\]
für jedes $n \in \{1, \dots, n - 1\}$. Für $n = p$ Primzahl und $\tau \in S_p$
Transposition, $\sigma \in S_p$ p-Zykel ist $S_p = \langle {\tau, \sigma}
\rangle$. \\
\textit{Beweis} $\dots$

\paragraph{Definition 5.52}
Für jedes $\sigma \in S_n$ heißt
\[
  sgn(\sigma) := \prod_{i < j}\frac{\sigma(i) - \sigma(j)}{i - j}
\]
das \textit{Signum} von $\sigma$.

\paragraph{Satz 5.53}
\begin{enumerate}
\item Das Signum definiert einen Gruppenhomomorphismus
  \[
    sgn: S_n \to (\{\pm 1\}, \circ),~\sigma \mapsto sgn(\sigma)
  \]
\item $sgn((x_1, \dots, x_k)) = -1^{k-1}$
\end{enumerate}
\textit{Beweis} $\dots$

\paragraph{Definition 5.54}
$A_n := ker(sgn) \in S_n$ heißt \textit{alternierende} Gruppe.

\section{Ringe und Module}
\label{sec:ringeundmodule}

\paragraph{Definition 6.1} 
Ein \textit{Ring} ist ein Tupel $(R, +, \cdot, 0, 1)$ wobei $R$ eine Menge, $+$
die Addition, $\cdot$ die Multiplikation, $0$ das neutrale Element der Addition
und $1$ das neutrale Element der Multiplikation ist, mit folgenden Eigenschaften:
\begin{itemize}
\item $(R, +, 0)$ ist eine abelsche Gruppe
\item $(R, \circ, 1)$ ist ein Monoid
\item $\forall a,b,c \in R: a \cdot (b + c) = a \cdot b + a \cdot c$
\end{itemize}
Ein Ring heißt textit{kommutativ}, falls $(R, \circ, 1)$ kommutativ ist.

\paragraph{Bemerkung 6.2}
$0$ und $1$ sind in einem Ring eindeutig definiert.

\paragraph{Beispiele 6.3}
\begin{enumerate}
\item $(\mathbb{Z}, +, \circ)$ ist ein Ring.
\item $(\mathbb{Q}, +, \circ)$ ist ein Ring.
\item $(\mathbb{K}, +, \circ)$ ist ein Ring, wobei $K$ ein Körper ist.
\item Seien $X$ eine Menge und $Funkt(X, \mathbb{R}) := \{f: X \to \mathbb{R}\}$
  dann ist $(Funkt(X, \mathbb{R}), +, \cdot)$ ein Ring.
\item $R = \{0\}$ mit  $0 = 1$ ist ein Ring.
\end{enumerate}

\paragraph{Lemma 6.4}
Ein Ring mit $0 \neq 1$ heißt $Divisionsring$ falls $\forall x \in R \backslash
\{0\}: x \cdot x^{-1} = 1$ gilt.
$R^\star := R \backslash \{ 0 \}$ ist eine Gruppe.
Ein Körper ist ein kommutativer Divisionsring.

\paragraph{Bemerkung 6.5}
Die Definition 6.4 ist äquivalent zur ersten Definition eines Körpers.

\paragraph{Definition 6.6}
Ein \textit{Ringhomomorphismus} $f: R \to S$ ist eine Abbildung mit dem
folgenden Eigenschaften:
\begin{itemize}
\item $\forall a,b \in R: f(a) + f(b) = f(a + b)$
\item $\forall a,b \in R: f(a) \cdot f(b) = f(a \cdot b)$
\item $f(1_R) = 1_S$
\end{itemize}
Punkt 3 folgt nicht automatisch weil $\cdot$ ein Monoid ist und keine Gruppe.

\paragraph{Beispiele 6.7}
\begin{itemize}
\item $\mathbb{Z} \subset \mathbb{Q},~a \mapsto a$
\item $Funk(X, \mathbb{R})$ ist ein Ring (Bsp.: 6.3.4)
  \[
    S = \mathbb{R}~~P \in X~~R \to S,~f \mapsto f(p)~\text{Evulationsabb.}
  \]
  ist ein Ringhomomorphismus.
\end{itemize}

\paragraph{Definition 6.8}
Sei $R$ ein Ring. Eine Teilmenge $S \subset R$ heißt \textit{Unterring}, falls:
\begin{itemize}
\item $S \subset R$ ist Untergruppe bzgl. Addition.
\item $S \cdot S \subset S$
\item $1_R \in S$
\end{itemize}

\paragraph{Beispiele 6.9}
\begin{enumerate}
\item $\mathbb{Z} \subset \mathbb{Q} \subset \mathbb{R} \subset \mathbb{C}$
\item In $(\mathbb{Z}, +, \cdot)$ gibt es keinen Unterring (außer $\mathbb{Z}$),
  denn alle  Untergruppe sind in der Form $n \mathbb{Z} \subset \mathbb{Z}$ aber
  $1_R \notin n \mathbb{Z}$
\item $\dots \subset C'(\mathbb{R}, \mathbb{R}) \subset Funkt(\mathbb{R},
  \mathbb{R})$
\end{enumerate}

\paragraph{Definition 6.10}
Sei $R$ ein Ring und $I \subset R$ eine Untergruppe bzgl. Addition.
\begin{itemize}
\item $I$ heißt \textit{Linksideal} $\Leftrightarrow R \cdot I \subset I$
\item $I$ heißt \textit{Rechtsideal} $\Leftrightarrow$ $I \cdot R \subset I$
\item $I$ heißt \textit{Zweiseitigesideal} $\Leftrightarrow I$ Rechts- und
  Linksideal
\end{itemize}
Beachte: Nenne Zweiseitigesideal, Ideal.

\paragraph{Bemerkung 6.11}
Sei $I$ Rechts-, Links- oder Zweiseitigesideal.
\begin{itemize}
\item $I \subset R ~\text{ist Unterring} \Leftrightarrow I = R$
\item Falls $R$ kommutativ ist, sind alle drei Begriffe äquivalent.
\end{itemize}

\paragraph{Beispiele 6.12}
\begin{enumerate}
\item Sei $R$ Ring $I = \{ 0 \}$ ist Zweiseitigesideal, genauso wie $I = R$.
\item In einem Körper $K$ gibt es keine Ideale $I \subset K~~I \neq \{ 0 \} \neq
  \{ K \}$
\item $R = \mathbb{Z}$ dann sind alle Ideal der Form $n \mathbb{Z} = I \subset
  \mathbb{Z}$
\item Sei $X$ Menge $R = Funkt(\mathbb{R}, \mathbb{R})$ und $P \in X$ mit $I =
  \{ f \in R ~|~ f(P) = 0\}$ ist Untergruppe ($I$ ist Ideal).
\end{enumerate}

\paragraph{Definition 6.13}
Sei $f: R \to S$ ein Ringhomomorphismus mit
\begin{enumerate}
\item $ker(f) = \{ a \in R ~|~ f(a) = 0\} \subset R$
\item $im(f) = \{ b  \in S ~|~ b = f(a), a \in R \} \subset S$
\end{enumerate}

\paragraph{Lemma 6.14}
\begin{enumerate}
\item $ker(f)$ ist zweiseitiges Ideal.
\item $im(f)$ ist ein Unterring.
\end{enumerate}

\textit{Beweis}
%\begin{enumerate}
%\item $f(a + b) = f(a) + f(b)~a,b \in ker(f) \Rightarrow a + b \in ker(f)
%  \Rightarrow \text{Untergruppe}$ \\
%  $0 \in ker(f),~~ a \in ker(f) \Rightarrow -a \in ker(f)$ \\
%  $f(a \cdot b) = \underbrace{f(a) \cdot \underbrace{f(b)}_{=0}}_{= 0}$
%\item $ im(f) \subset S$ Unterring \\
%  Untergruppe, $im(f) \cdot im(f) \subset im(f), 1_S \in im(f)$ \\
%  $a_1, a_2 \in im(f) \exists b_1, b_2 \in R: f(b_1) = a_1$
%\end{enumerate}

\paragraph{Lemma 6.15}
Sei $f: R \to S$ Ringhomomorphismus. Dann gilt
\begin{enumerate}
\item $f$ injektiv $\Leftrightarrow ker(f) = \{0\}$
\item $f$ surjektiv $\Leftrightarrow im(f) = S$
\end{enumerate}
\textit{Beweis} \\
1.  $f(a) = f(b) \Leftrightarrow a = b$ \\
$f(a) = f(b) \Leftrightarrow f(a) - f(b) = 0$ \\
Aber $f(a) - f(b) = f(a - b)$ Deswegen:
$a - b \in ker(f)$
\\\\
$\Rightarrow$ $f$ injektiv $\Rightarrow a = b \Rightarrow 0 \in ker(f)$
\\\\\\
$\Leftarrow$ \\
$a - b = 0 \Leftrightarrow a = b$\\
Beweis von 2. ist klar
$\hfill \square$

\paragraph{Bemerkung 6.16}
$S \subset R$ wobei $S$ Unterring. Betrachte Abbildung
\[
  S \to R,~a \mapsto a
\]
die einen Ringhomomorphismus darstellt. D.h. jeder Unterring kann als Bild eines
Ringhomomorphismus betrachtet werden.

\paragraph{Definition 6.17}
Sei $I \in R$ zweiseitiges Ideal. Die additive Gruppe von $R$ ist abelsch
$\Rightarrow (I, +, 0) \subset (R, +, 0)$ Normalteiler.
\[
  R/I ~\text{Faktorgruppe}~~((a+I)(b+I) = (a+b) + I) ~~(**)
\]
Definiere Multiplikation auf $R/I$ durch:
\[
  (a + I) \cdot (b + I) := ab + I ~~(*)
\]

\paragraph{Lemma 6.17}
Die Multiplikation in Definition 6.17 ist wohldefiniert. \\
\textit{Beweis} \\
$a + I, b + I~~ x \in a + I, y \in b + I$ \\
Müssen zeigen, dass $x \cdot y \in ab + I$ \\
$x \in a + I \Leftrightarrow x = a + u, u \in I $ \\
$y \in b + I \Leftrightarrow y = a + v, v \in I $ \\
$xy = (a + u)(b + v) = ab + av + au + ub + uv$ \\
$av + au + ub + uv \in I$ weil $I$ zweiseitiges Ideal.
$\hfill \square$

\paragraph{Theorem 6.18}
\begin{enumerate}
\item $R/I$ mit $(*)$ und $(**)$ ist ein Ring.
\item $R \to R/I,~a \mapsto a + I$ ist ein Epimorphismus mit $ker = I$.
\end{enumerate}
\textit{Beweis} \\
Zu 1.: \\
Aus Def 7.1 folgt, dass $R/I$ mit $(**)$ eine abelsche Gruppe ist. \\
$R/I$ mit $(*)$ ist ein Monoid ? \\
Assoziativität: $((a + I)(b + I))(c + I) = (ab + I)(c + I) = (ab)c + I = a(bc) +
I = (a + I)(bc + I) = (a + I)((b + I)(c + I))$ \\
Eins-Element: $(1 + I)(a + I) = a + I$ und $(a + I)(1 + I) = (a + I)$ \\
$1 + I$ die Eins. \\
$(**) und (*)$ sind distributiv: $((a + I) + (b + I))( c+ I) = (a + I)(c+ I) +
(b + I)(c + I)$
Zu 2.: \\
klar
$\hfill \square$

\paragraph{Beispiel 6.19}
\begin{enumerate}
\item $R = (\mathbb{Z}, +, \cdot),~~ I = n\mathbb{Z},~~ \mathbb{Z}/n\mathbb{Z}$
  hat $n$ Elemente.
\end{enumerate}

\paragraph{Theorem 6.20}
Sei $f: R \to S$ ein Ringhomomorphismus. Dann gibt es einen natürlichen
(kanonischen) Isomorphismus
\[
  \overline{f}: R/ker(f) \overset{\simeq}{\to} im(f)
\]
\textit{Beweis} $R \to im(f) \subset S$ und $R \overset{f}{\to} S$. Schicke $a
\in R$ auf $a \mapsto a + I \in R/I$. \\
$\overline{f}(a + I) := f(a)$ Wohldef.? Ja. Übung \\
$\overline{f}$ surjektiv. OK \\
$\overline{f}$ injektiv $\Leftrightarrow ker(\overline{f}) = \{0\} = \{a + I\}$ \\
$f(a + ker(f)) = f(a) = 0 \Rightarrow a \in ker(f)$ \\
Wenn $a \in ker(f) \Rightarrow a + ker(f) \in R/ker(f)$ ist die Null
$\Rightarrow \overline{f}$ injektiv.
$\hfill \square$

\paragraph{Korollar 6.21}
Sei $f: R \to S$ ein Ringhomomorphismus.
\begin{enumerate}
\item $f$ Epimorphismus $\Rightarrow S \simeq R/ker(f)$
\item $f$ lässt sich schreiben als
  \[
    R \overset{f}{\to} S, ~ R \to R/ker(f) \overset{\overline{f}}{\to} im(f)
    \subset S
  \]
\end{enumerate}
\textit{Beweis} \\
1. folgt sofort aus Theorem 6.20

\paragraph{Satz 6.22}
(\textit{Korrespodenzsatz für Ideale}) Sei $R$ ein Ring und $I \subset R$ ein
zweiseitiges Ideal. Dann induziert die kanonische Projektion
\[
  \psi: R \to R/I
\]
eine Bijektion zwischen Linksideale in $R/I$ und Linksideale in $R$, die $I$
enthalten. Analog geht es für Rechtsideale und Zweiseitigeideale. \\
\textit{Beweis} \\
Übungsblatt. Sehr ähnlich zu Satz 5.23.

\subsection{Kommutative Ringe}
\label{subsec:kommutativeringe}

\paragraph{Beispiel 6.23} (\textit{Polynomringe})
\begin{enumerate}
\item Sei $R$ kommutativer Ring (z.B. $R = \mathbb{Z}$) \\
  $x$ eine Variable (= Buchstabe)
  Ein Polynom in $x$ über $R$ ist ein Ausdruck der Form
  \[
    a_0x^n + a_1x^{n-1} + \dots + a_{n-1}x + a~~~a_i \in R~n \in
    \mathbb{Z}_{\geq 0}
  \]
  Addition: $P, Q$ Polynome
  \[
    P = a_nx^n + \dots a_{1}x+a_0 ~~
    Q = b_mx^m + \dots + b_{1}xb_0
  \]
  \[
     P + Q = \sum (a_i + b_i)x^i
   \]
   Multiplikation:
   \[
     P \cdot Q = \sum_{k = 0}^n \left( \sum_{i = 0}^k a_ib_{k - i} \right)x^k
   \]
   Die Eins ist $P = 1$ und die Null ist $P = 0$. \\
   $R[x] = \{\text{Polynome in $x$ über $R$}\}$ ist ein kommutativer Ring.

 \item Variablen $x_1, \dots, x_n$
   \[
     R[x_1, \dots, x_n] := \left(R[x_1, \dots, x_n]\right)[x_n]
   \]
   Explizit: Ein Element in $R[x_1, \dots, x_n]$ ist ein Ausdruck der Form
   \[
     P(x_1, \dots , x_n) = \sum_{I = (i_1, \dots, i_n)} a_I~x_i^{i_1} \dots x_n^{i_n}
   \]
   Warnung: $R = K$, $K$ ist Körper $P(x) \in K[x]$
   \[
     a \in K \mapsto P(a) \in K
   \]
   bekomme eine Abbildung
   \[
     \overset{\sim}{P}: K \to K
   \]
   Es kann passieren, dass
   \[
     \overset{\sim}{P} = \overset{\sim}{Q} ~\text{für} ~ P \neq Q
   \]
\end{enumerate}

\paragraph{Definition 6.24}
\begin{enumerate}
\item Seien $I,J$ Ideale
  \[
    I + J := \{a + b \in R ~|~ a \in I~a \in J\}
  \]
\item
  \[
    I \cdot J := \{a_1b_1 + \dots + a_nb_n ~|~ a_i \in I~ b_i \in J~n \geq
    \mathbb{Z}_{\geq 1}\}
  \]
\item $M \subset R$ Teilmenge
  \[
    (M) := \{r_1a_1 + \dots + r_na_n ~|~ r_i \in R~ a_i \in M ~ n \geq
    \mathbb{Z}_{\geq 1}\} ~\text{Ideal}
  \]
  Ideal erzeugt von $M$.
\item $M = \{m_1, \dots, m_n\}$ danns schreibt man
  \[
    (m_1, \dots, m_n) := (M)
  \]
  Ideal erzeugt von $(m_1, \dots, m_n)$.
\item $M = \{m\}$ heißt $(M) = (m)$ \textit{Hauptideal}
  \[
    (m) = \{rm \in R ~|~ r \in R\}
  \]
\end{enumerate}

\paragraph{Beispiel 6.25}
\begin{enumerate}
\item $(n) = n\mathbb{Z} \subset \mathbb{Z}$ Hauptideal.
\end{enumerate}

\paragraph{Definition 6.26}
Sei $I \subset R$ ein Ideal.
\begin{enumerate}
\item $I$ heißt maximal $\Leftrightarrow$ Es gibt kein echtes Ideal, das $I$
  enthält.
\item $I$ prim $\Leftrightarrow$ (wenn $x \cdot y \in I \Rightarrow x \in I$
  oder $y \in I$)
\end{enumerate}

\paragraph{Lemma 6.27}
Jedes maximale Ideal ist prim. \\
\textit{Beweis} \\
maximal $\Rightarrow$ prim. $I \subset R$ maximales Ideal $x,y \in R$ mit $xy
\in I$ \\
Nehmen wir an, dass $x \notin I \overset{?}{\Rightarrow} y \in I$
\[
  I \subset I + (x) = \{a + bx ~|~ a \in I~b \in R\} \subset R
\]
Da $I$ maximal $\Rightarrow I + (x) = R$
\[
  1 = a + bx ~~ a \in I ~ b \in R
\]
mit $y$ multiplizieren
\[
  y = \underset{\in I}{ay} + \underset{\in I}{bxy} \in I
\]
$\hfill \square$

\paragraph{Lemma 6.28}
$I \subset R$ ist maximal $\Leftrightarrow R/I$ ist Körper \\
\textit{Beweis} \\
$S$ ist ein Körper  $\Leftrightarrow$ Es gibt keine nichttriviale Ideale. \\
$R/I$ ist Körper $\Leftrightarrow$ Es gibt keine Ideale außer $(0)$ und $R/I$.\\
Nach Korrespodenzsatz für Ideale: Ideale in
$R/I \underset{\text{1 zu 1}}{\leftrightarrow} I \subset J
\subset R$ folgt die Aussage.
$\hfill \square$

\paragraph{Beispiel 6.29}
$R = \mathbb{Z}$ Alle Ideale sind von der Form $(n), n \in \mathbb{Z}$ \\
Welche sind maximal?
\[
  (n) ~\text{maximal} \Leftrightarrow n ~\text{ist eine Primzahl}
\]
$(p) \subset \mathbb{Z} ~~ \mathbb{Z}/(p) =: \mathbb{F}_{2}$ der endlichen Körper
mit $p$ Elementen.

\paragraph{Definition 6.30}
Sei $R$ ein kommutativer Ring
\begin{enumerate}
\item $0 \neq x \in R$ heißt \textit{Nullteiler}
  $\Leftrightarrow \exists 0 \neq y \in R$ sodass $xy = 0$
\item $R$ heißt Integritätsbereich $\Leftrightarrow$
  Es gibt keine Nullteiler und $0 \neq 1$
\end{enumerate}

\paragraph{Beispiel 6.31}
\begin{enumerate}
\item $R = \mathbb{Z}$ keine Nullteiler  $\Rightarrow$ Integritätsbereich
\item $R = \mathbb{Q}[x]$ keine Nullteiler $\Rightarrow$ Integritätsbereich
\item $R = \mathbb{Q}[x]/\left(x^2\right)$ ~~
  ($(x + I)(x + I) = \underset{\in I}{x^2} + I = I$) \\
  $\Rightarrow$ Nullteiler, kein Integritätsbereich.
\item $(0) \in R$ ist prim $\Leftrightarrow R$ ist Integritätsbereich \\
  \textit{Beweis} \\
  Übung
\end{enumerate}

\paragraph{Lemma 6.32}
Sei $I \subset R$ ist prim $\Leftrightarrow R/I$ ist Integritätsbereich. \\
\textit{Beweis} \\
$x + I \in R/I$ und $x + y \in R/I$ ~~ $(x+I)(y+I) = xy + I$ \\
\grqq $\Rightarrow$ \grqq \\
Nehmen wir an, dass $I$ prim ist.
\[
  (x + I)(y + I) = I \Leftrightarrow xy \in I \Rightarrow (x \in I \vee y \in I)
  \Leftrightarrow (x + I) = I ~oder~ y + I = I
  \Rightarrow R/I ~\text{Integritätsbereich}
\]
\grqq $\Leftarrow$ \grqq \\
$R/I$ Integritätsbereich $\Rightarrow$ $I$ prim.
\[
  (x + I)(y + I) = I \Rightarrow x \in I ~oder~ y \in I
\]
\hfill $\square$

\paragraph{Definition 6.33}
Sei $M$ eine Menge
\begin{enumerate}
\item Eine \textit{partielle Ordnung} auf $M$ ist eine Relation $\leq$, die reflexiv,
  transitiv und antisymmetrisch ist.
\item Eine Ordnung heißt \textit{total} $\Leftrightarrow \forall x,y \in M: ~
  \text{entweder} ~ x \leq y ~\text{oder}~ y \leq x$.
\item $x \in M$ heißt \textit{größtes Element} $\Leftrightarrow \forall a \in M:
  a \leq x$.
\item $x \in M$ heißt \textit{maximal} $\Leftrightarrow x \leq a$
  impliziert $x = a$.
\item $x \in M$ heißt \textit{obere Schranke} für eine Teilmenge $N \subset M
  \Leftrightarrow \forall a \in N: a \leq x$.
\end{enumerate}

\paragraph{Beispiel 6.34}
\begin{enumerate}
\item $\mathbb{R}, ~ \leq$
\item $X$ eine Menge $\rightsquigarrow M = $ die Menge von Teilmengen von $X$. Als
  $\leq$ nehmen wir die Inklusion von Teilmengen.
\end{enumerate}

\paragraph{Bemerkung 6.35}
\begin{enumerate}
\item $\exists$ höchstens ein größtes Element.
\item Wenn ein größtes Element existiert $\Rightarrow \exists$ nur ein maximales
  Element. \\
  \textit{Beweis} \\
  $x$ größtes Element $z$ ein maximales Element $\Rightarrow x = z$ \\
  $z \leq x$ (da $x$ größtes Element ist).
  $\overset{Def.~ 6.33.4}{\Rightarrow} x = z$
\item Meistens gibt es kein größtes Element aber viele maximale Elemente.
\end{enumerate}

\paragraph{Lemma 6.36}
\textit{(Lemma von Zorn)} Sei $M$ eine partiell geordnete Menge, in der jede
total geordnete Teilmenge (= eine Kette) eine obere Schranke hat.
Dann enthält $M$ mindestens ein maximales Element. \\
\textit{Beweis} \\
Bosch, §3.4, Lemma 5

\paragraph{Lemma 6.37}
Sei $I \in R$ ein echtes Ideal. Dann existiert ein maximales Ideal $m \in R$,
sodass $I \in m \in R$. \\
\textit{Beweis} \\
$M = \{J \underset{\neq}{\subset} R ~|~ I \subset J \} = $ die Menge aller echten
Idealen in $R$, die $I$ enthalten, $J_1 \leq J_2 \Leftrightarrow J_1 \subset
J_2$ Kette von Idealen $J_1 \subset J_2 \subset \dots$ Wir müssen zeigen, dass
es ein echtes Ideal $J \in I \in R$ gibt sodass $\forall i : J_i \subset J$. \\
$J := \bigcup_i J_i \subset R$
\begin{enumerate}
\item $J$ ist Ideal ?: \\
  $x,y \in J \Rightarrow \exists a: x \in J_a ~\text{und}~ \exists b: b \in J_b$ \\
  $c = max(a,b) \Rightarrow x,y \in J_c \Rightarrow x + y \in J_c \subset J$. \\
  --- \\
  $x \in J \Rightarrow x \in J_a$ \\
  $x \cdot r \in J$: $x \cdot r \in J_a \subset J$.
\item $J \neq R$ ?: \\
  Sei $J = R \Rightarrow 1 \in J \Rightarrow \exists a: 1 \in J_a \Rightarrow J
  = R ~~\lightning$
\end{enumerate}

\paragraph{Beispiel 6.38} $\mathbb{Z}$ welche Ideale sind maximal? \\
maximale Ideale = $(p) = p \mathbb{Z} ~p$ eine Primzahl. \\
maximale Ideale $\subset$ Primideale $= ~\text{maximale Ideale}~ \cup \{(0)\}$.

\paragraph{Korollar 6.39}
Jedes nicht invertierbare Element in $R$ liegt in einem maximalen Ideal. \\
\textit{Beweis} \\
$z \in I = (z) = Rz \neq R \hfill \square$

\subsection*{Hauptidealringe}
\label{subsec:hauptidealbereiche}

\paragraph{Wissen} ~\\
$R$ kommutativer Ring $I \subset R$ heißt Hauptideal $\Leftrightarrow \exists x
\in R: I = (x) := \{ax ~|~ a \in R\}$

\paragraph{Definition 6.40}
\begin{enumerate}
\item $R$ heißt Hauptidealring $\Leftrightarrow 0 \neq 1$ und jedes Ideal ein
  Hauptideal ist.
\item \textit{Hauptidealbereich} = Hauptidealring + Integritätsbereich
\end{enumerate}

\paragraph{Beispiel 6.41}
\begin{enumerate}
\item $R = \mathbb{Z}$ Hauptidealbereich.
\end{enumerate}

\paragraph{Definition 6.42}
Man sagt, dass in einem Ring $R$ eine Division mit Rest existiert, wenn es eine
Funktion $\phi: R \to \mathbb{Z}_{\geq 0}$, soadss
\begin{enumerate}
\item $\phi^{-1}(0) = \{0\}$
\item $\forall a,b \in R, b \neq 0, \exists q,r \in R: a = qb + r$
  und $\phi(r) < \phi(b)$ ($q$ Quotient, $r$ Rest)
\end{enumerate}

\paragraph{Beispiel 6.43}
\begin{enumerate}
\item $R = \mathbb{Z}, \phi(n) = |n|$
\item $R = K[X]$ wobei $K$ Körper ist. \\
  $\phi(P(X)) := deg(P(X)) + 1$ und $deg(0) := -1$
\end{enumerate}

\paragraph{Lemma 6.44} \textit{(Polynomdivision)}
Seien $K$ ein Körper und $A(x),B(x) \neq 0 \in K[x]$. Dann existieren eindeutige
Polynome $Q(x), R(x) \in K[x],$ soadss
\[
  A(x) = Q(x)B(x) + R(x)
\]
mit
\[
  deg(R(x)) < deg(B(x))
\]
\textit{Beweis} \\
Übungsblatt

\paragraph{Lemma 6.45}
Ein Ring mit Division mit Rest ist ein Hauptidealring. \\
\textit{Beweis} \\
$(0) \neq I \subset R$, $0 \neq b \in I$ mit der Eigenschaft
\[
  \phi(b) = min\{\phi(x) ~|~ 0 \neq x \in I\}
\]
$\overset{?}{\Rightarrow}$ $I = (b)$ \\
$a \in I \Rightarrow a = qb + r ~~ \phi(r) < \phi(b) \Rightarrow r = 0
\Rightarrow a = qb \hfill \square$

\paragraph{Definition 6.46}
Sei $R$ ein Integritätsbereich.
\begin{enumerate}
\item $d \in R$ heißt Teiler von $a \in R \Leftrightarrow a = dc ~ c \in R$.
  Man schreibt $d|a$ (d teilt a).
\item $d \in R$ heißt größer gemeinsamer Teiler von $a,b \in R \Leftrightarrow$
  \begin{itemize}
  \item $d|a$ und $d|b$
  \item $\forall d' \in R: d'|a$ und $d'|b$ gilt $d'|d$
  \end{itemize}
  Größter gemeinsamer Teiler ist bis auf Multiplikation mit einem invertierbaren
  Element eindeutig definiert.
  \textbf{Notation:} $gcd(a,b)$
\end{enumerate}

\paragraph{Lemma 6.47}
Sei $R$ ein Hauptidealbereich $\Rightarrow \forall a,b \in R: \exists ~ \gcd(a,b)$
\\
\textit{Beweis} \\
$a,b \in R$ \\
$ I = (a,b) = aR + bR$ \\
$\exists d \in R: I = (d) = dR$ \\
$d \overset{?}{=} \gcd(a,b)$ \\
$d|a$ ? $a \in I = (d) \Rightarrow a = dx, x \in R \Rightarrow d|a$ für $d|b$ analog.
\\
$d'|a ~~ a = d'x'$ \\
$d'|b ~~ b = d'y'$ \\
$\overset{?}{\Rightarrow} d'|d$ wegen \\
$d \in I ~~ d = \underset{d'x'}{a \alpha}
+ \underset{d'y'}{b \beta} , \alpha, \beta \in R$ \\
$d = d'(x' \alpha + y' \beta) \hfill \square$

\paragraph{Bemerkung 6.48}
Sei $R$ Hauptidealbereich.
\[
  d = \gcd(a,b) \Rightarrow d = a\alpha + b\beta, mit \alpha,\beta \in R
\]

\paragraph{Definition 6.49}
Ein Ring $R$ heißt
\begin{enumerate}
\item \textit{noethersch} $\Leftrightarrow$ Jede aufsteigende Kette von Idealen
  stationär wird.
\item \textit{artinsch} $\Leftrightarrow $ Jede absteigende Kette von Idealen
  stationär wird.
  \begin{itemize}
  \item aufsteigende Kette: $I_1 \subset I_2 \subset \dots$
  \item stationär $\Leftrightarrow \exists n: I_j = I_n$ f.a. $j \geq n$
  \item absteigende Kette: $\dots \subset I_2 \subset I_1$
\end{itemize}
\end{enumerate}

\paragraph{Beispiel 6.50}
$R = \mathbb{Z} ~~ (n_1) \subset (n_2) \subset \dots$ \\
$n_i \in \mathbb{Z}$ mit $n_{i+1}|n_i$ \\
\begin{itemize}
 \item $\Rightarrow$ jede aufsteigende Kette stationär wird
\item $\Rightarrow \mathbb{Z}$ ist noethersch.
\end{itemize}
\textbf{Frage:} Ist $\mathbb{Z}$ noethersch? Nein!

\paragraph{Lemma 6.51}
Hauptidealringe sind noethersch. \\
\textit{Beweis} \\
\[
  I_1 \subset I_2 \subset \dots
\]
aufsteigende Kette von Idealen.
\[
  I = (d) = \bigcup_{j = 1}^\infty I_j, ~~ d \in R ~~ \text{Ideal}
\]
\[
  d \in I \Rightarrow \exists n: d \in I_n \Rightarrow  I \subset I_n
  \Rightarrow I_j = I_n ~~ \text{f.a.} ~ j \geq n 
\]
\hfill $\square$

\subsection{Faktorielle Ringe}
\label{subsec:faktorielleringe}

\paragraph{Definition 6.52}
\begin{enumerate}
\item $u \in R$ heißt Einheit $\Leftrightarrow u$ hat ein multiplikatives Inverses.
  \\
  Die Einheiten bilden eine Gruppe bzgl. Multiplikation. Diese Gruppe bezeichnet
  man mit $R^*$.
\item $a \in R$ heißt irreduzibel $\Leftrightarrow a \notin R^*$ und $(a = bc$
  impliziert $b \in R^*$ oder $c \in R^*$).
\item $x \in R$ heißt prim $\Leftrightarrow$ das Ideal $(x)$ ist ein Primideal.
  $\Leftrightarrow (x|ab \Rightarrow x|a ~oder~ x|b)$
\end{enumerate}

\paragraph{Bemerkung}
$R$ Integritätsbereich, dann ist jedes Primelement irreduzibel.

\paragraph{Lemma 6.61}
$R$ Integritätsbereich. Für ein Element $a \in R$ habe man Zerlegung
\[
  a = p_1 \dots p_r = q_1 \dots q_s
\]
wobei $p_i$ Primelemente sind und $q_i$ irreduzibel. Dann gilt $r = s$ und nach
evntl. Umnummerierung der $q_j$ ist $p_i$ assoziert zu $q_i$
f.a. $i = 1, \dots, r$. \\
\textit{Beweis}
\[
  a = p_i \dots p_r = q_1 \dots q_s
\]
also
\[
  p_1|q_1, \dots q_s \Rightarrow \exists j: p_1|q_j
\]
Nach Umnummerierung, können wir $j = 1$ annehmen. D.h. $p_1|q_1$
\[
  p_1|q_1 \Rightarrow q_1 = p_1\underbrace{u}_{\in R^*}
  \Rightarrow p_2 \dots p_r = uq_2 \dots q_s
\]
Beweis im Allgemeinen per Induktion.
\hfill $\square$

\paragraph{Lemma 6.62}
Sei $R$ ein Integritätsbereich. Dann ist äquivalent:
\begin{enumerate}
\item Jedes Element $a \in R\backslash (R^*\cup \{0\})$ lässt sich eindeutig
  (bis auf die Reihenfolge) als Produkt von irreudziblen Elementen schreiben.
\item Jedes $a \in R\backslash (R^*\cup \{0\})$ lässt sich eindeutig
  (bis auf die Reihenfolge) als Produkt von Primelementen schreiben. 
\end{enumerate}
\textit{Beweis}
\paragraph{1. $\Rightarrow$ 2.} 
Es genügt zu zeigen, dass jedes irreudzible Element ein Primelement ist.
$a$ irreduzibel $\Rightarrow a$ prim
\[
  a|xy \overset{?}{\Rightarrow} a|x ~oder~ a|y
\]
\[
  x = x_1 \dots x_n ~~ y = y_1 \dots y_m
\]
Wobei $x_i,y_i$ irreduzibel sind.
\[
  \Rightarrow a|x_1 \dots x_n y_1 \dots y_m
\]
Wegen Eindeutigkeit
\[
  \Rightarrow \exists x_i ~oder~ y_i: a ~\text{a ist assoziativ zu}~x_i,y_i
\]
\paragraph{2. $\Rightarrow$ 1.}
Übung.

\paragraph{Definition 6.63}
Ein Integritätsbereich $R$ heißt faktoriell, wenn die äquivalenten Bedingunen
von Lemma 6.62 erfüllt sind.

\paragraph{Bemerkung 6.64}
$R$ faktoriell. $a$ irreduzibel $\Leftrightarrow a$ ist Primelement. 

\paragraph{Beispiel 6.65}
\begin{enumerate}
\item $R = \mathbb{Z}$
  \[
    R^* = \{ \pm 1\}
  \]
\item $ n \in \mathbb{Z}$ ist irreduzibel $\Leftrightarrow |n|$ ist prim.
\item $R$ bel. kommutativer Ring. $a \in R ~~ u \in R^* \Rightarrow (a) = (au)$
\end{enumerate}

\paragraph{Lemma 6.66}
Sei $R$ ein Integritätsbereich und $a,b \in R$
\[
  (a) = (b) \Leftrightarrow b = au ~~ u \in R^*
\]
\textit{Beweis} \\
\grqq $\Leftarrow$ \grqq ~~ Klar. \\
\grqq $\Rightarrow$ \grqq \\
O.B.d.A. $b \neq 0$ dann gilt:
\[
  (a) = (b) \Rightarrow b \in (a) \Rightarrow b = ax ~~ x \in R
\]
\[
  (a) = (b) \Rightarrow a \in (b) \Rightarrow a = by ~~ y \in R
\]
$b = byx \Leftrightarrow b(1 - yx) = 0 \overset{b~=~0~+
  ~\text{Integritätsbereich}}{\Rightarrow} 1 - yx = 0$
\hfill $\square$

\paragraph{Lemma 6.67}
Sei $R$ ein Hauptidealbereich, $p in R$ ein irreduzibeles Element und
$p|ab, ~~ a,b \in R$. Dann gilt:
\[
  p|a ~\text{oder}~ p|b
\]
\textit{Beweis} \\
Nehmen wir an, dass $p \nmid a$ $\overset{?}{\Rightarrow} p|b$ \\
$\Rightarrow \gcd(a, p) = 1 \Rightarrow 1 = ax + py
\overset{b}{\Rightarrow} b = bax + bpy = p(dx + by) \Rightarrow p|b$ \hfill $\square$

\paragraph{Lemma 6.68}
Sei $R$ ein Integritätsbereich und $a \in R$. Dann gilt:
\[
  (a) ~ \text{ist prim} \Rightarrow a ~ \text{ist irreduzibel}
\]
\textit{Beweis} \\
Nehmen wir an, dass $a$ nicht irreduzibel ist.
\[
  \Rightarrow a = bc ~ \text{mit} ~ b,c \in R \backslash R^*
\]
Übung.

\paragraph{Definition 6.69}
\begin{enumerate}
\item Ein Element $0 \neq a \in R$ besitzt eine eindeutige Zerlegung in
  irreduzibele Faktoren, wenn
  \[
    a = u \cdot \bigsqcap_{i = 1}^n p_i
  \]
  mit $u \in R^*, p_i$ irreduzibel und für jede solche Darstellung
  \[
    a = u' \cdot \bigsqcap_{i = 1}^m p'_i
  \] $(*)$
  mit $u' \in R^*, p_i$ irreduzibel gilt:
  \begin{enumerate}
  \item $m = n$
  \item $P_i = v_ip'_i ~~ v_i \in R^*$
  \end{enumerate}
\item $R$ heißt faktorieller Ring  $\Leftrightarrow$
  \begin{itemize}
    \item $R$ ist Integritätsbereich.
    \item Jedes Element hat eine Zerlegung $(*)$
  \end{itemize}
\end{enumerate}

\paragraph{Beispiel 6.70}
\begin{enumerate}
\item $\mathbb{Z}$ faktoriell
\item $\mathbb{Z}\left[\sqrt{-5}\right]$ ist nicht faktoriell.
\end{enumerate}

\paragraph{Theorem 6.71}
$R$ Hauptidealbereich $\Rightarrow R$ ist faktoriell. \\
\textit{Beweis} \\
$a \in R\backslash (R^*\cup \{0\})$
\begin{enumerate}
\item Zu zeigen: $a$ hat einen irreduziblen Teiler. \\
  $a$ irreduzibel $\Rightarrow$ fertig. Falls $a$ reduzibel $\Rightarrow a =
  a_1b_1 ~~ a_1,b_1 \notin R^* \Rightarrow (a) \subset (a_1) = Ra$. Falls nun
  $a_1$ irreduzibel sind wir wieder fertig. Falls $a$ reduzibel $\Rightarrow
  a_1 = a_2b_2 ~~ a_2,b_2 \notin R^*$. Nach Lemma wird diese Kette stationär.
  $\Rightarrow a_n$ ist irreduzibel und $a_n|a$.
\item Zu zeigen: $a$ lässt sich als Produkt von irreduziblen Elementen
  schreiben. \\
  $a$ hat irreduziblen Teiler
  \[
    \Rightarrow a = a_1q_1
  \]
  $q_1$ irreduzibel. Es folgt wieder $a_1 \in R^*$ oder $a_1$ hat einen
  irreduziblen Teiler $\Rightarrow (a) \subset (a_1)$
  \[
    (a) \subset (a_1) \subset (a_2) \dots
  \]
  wird stationär $\Rightarrow a = n q_1 \dots q_n$
  
\item Zu zeigen: Produkt aus 2. ist eindeutig (bis auf die Assoziativität und
  Reihenfolge).
\end{enumerate}
\hfill $\square$

\paragraph{Korollar 6.72}
$\mathbb{Z}$ und $K[x]$ sind faktoriell.

\paragraph{Frage:} $K$ Körper. Ist $K$ faktoriell?
$x \in R$ Primelement $\Leftrightarrow (x)$ ist Primideal ?
Denke darüber nach!

\paragraph{Bemerkung 6.73}
Es gibt faktorielle Ringe, die keine Hauptidealbereiche sind
\[
  K[x_1, \dots, x_n] ~\text{ist faktoriell, aber kein Hauptidealbereich}~
  n \geq 2, K ~ Körper
\]

\paragraph{Theorem 6.74}
Sei $R$ ein faktorieller Ring. Dann ist auch der Polynomring $R[x]$ faktoriell.

\paragraph{Korollar 6.75}
$R$ faktorieller Ring $\Leftarrow R[x_1, \dots, x_n]$ ist faktoriell
(insbesondere kan $R$ ein Körper sein). \\
\textit{Beweis} \\
\[
  (R[x_1])[x_2] = R[x_1,x_2] \Rightarrow \text{die Aussage folgt}
\]
\hfill $\square$

\paragraph{Definition 6.76}
Sei $R$ ein Integritätsbereich. Betrachten wir die Menge aller Paare
\[
  M = \{(a,b) ~|~ a \in R ~ b \in R\backslash \{0\} \}
\]
Äquivalenzeralation: $(a,b) \sim (a', b') :\Rightarrow ab' = a'b$
\[
  Q(R) := M/\sim
\]
Die Äquivalenzklassen von $(a, b)$ bezeichnen wir mit $\frac{a}{b}$ sodass
\[
  \frac{a}{b} = \frac{a'}{b'} \Rightarrow ab' = ba'
\]
Man kann zeigen (Übung) dass $Q(R)$ mit
\[
  \frac{a}{b} + \frac{c}{d} = \frac{ad + cb}{bd} ~~\text{Addition}
\]
\[
  \frac{a}{b} \cdot \frac{c}{d} = \frac{ac}{bd} ~~\text{Multiplikation}
\]
ein Körper ist. Dieser Körper heißt Quotientenkörper. \\
$\exists$ eine natürliche Abbildung
\[
  R \to Q(R), a \mapsto \frac{a}{1}
\]
ist ein injektiver Ringhomomorphismus.

\paragraph{Beispiel 6.77}
\begin{enumerate}
\item $R = \mathbb{Z}, Q(R) = \mathbb{Q}$
  \[
    \frac{a}{b} \in \mathbb{Q} ~~ a,b \in \mathbb{Z} ~ b \neq 0
  \]
\item Rationale Funktionen
  \[
    F = \frac{A(x)}{B(x)} ~~ B(x) \neq 0 ~ A(x),B(x) \in K[x] ~ K ~\text{Körper}
  \]
\end{enumerate}

\paragraph{Definition 6.78}
\begin{enumerate}
\item Sei $R$ ein faktorieller Ring und $P$ ein Vertretersystem von Primelementen,
d.h. aus jeder Klasse von zueinander assoziativen Primelementen enthält $P$
genau eines. Dann $\forall a \in R\backslash\{0\}$ lässt sich in der Form
\[
  a = u \prod_{p\in P}p^{v_p(a)}
\]
wobei $u \in R^*, v_p(a) \in \Z_{\geq 0}, v_p(a)=0$ für fast alle
$p\in P$.
\item $Q(R) \supset R, \frac{a}{1} \leftarrow a$
  \[
    \forall \frac{a}{b} \in Q(R)\backslash\{0\}: \frac{a}{b} = u\prod_{p\in
      P}p^{v_p(\frac{a}{b})}
  \]
\item
  \[
    f = \sum_{i=0}^na_ix^i \in Q(R)[x]
  \]
  und $v_p(f):=\us{i}{\min}v_p(a_i)$
\end{enumerate}

\paragraph{Bemerkung 6.79}
\begin{enumerate}
\item $f \in Q(R)[x]$ liegt in $R[x] \iff v_p(f) \geq 0$ f.a. $p\in P$
\item $v_p: Q(R)^* \to \Z$ Gruppenhomomorphismus.
\end{enumerate}

\paragraph{Lemma 6.80} \textit{(Gauß)} Sei $R$ faktoriell, $p$ Primelement. Dann
gilt
\[
  v_p(fg)=v_p(f)+v_p(g) ~~f,g \in Q(R)[x]
\]
\textit{Beweis}
Die Aussage ist klar für konstante Polynome. O.B.d.A $f,g \neq 0~~f,g\in
R[x]$(mit den Nennern multiplizieren) und $v_p(f)=0=v_p(g)$ (mit dem GGT von
Koeff. teilen). Wir müssen also zeigen $v_p(fg)=0$
\[
  R[x] \longrightarrow (R/(p))[x],
  \us{a_i\in R}{\sum} a_ix^i=h \mapsto \s{h}=\sum \s{a_i}x^i
\]
$(R/(p))$ ist Integritätsbereich
\[
  ker(\phi) = \{h\in R[x]~|~v_p(h)>0\}
\]
\[
  \phi(f) \neq 0 \neq \phi(g) \Rightarrow \phi(fg) = \phi(f)\phi(g) \neq 0
  \Rightarrow v_p(fg)=0
\]
\hfill $\square$

\paragraph{Korollar 6.81}
$R$ faktorieller Ring.
\begin{enumerate}
  \item Es sei $h\in R[x]$ ein normiertes (Leitkoeffizent = $1$) Polynom und
    $h=f\cdot g$ eine Zerlegung in normierte Polynome $f,g \in Q(R)[x]$. Dann gilt
    \[
      f,g \in R[x]
    \].
  \item Sei $h \in R[x]$ und $h=f\cdot g$ eine Zerlegung mit $f\in R[x]$
    primitiv
    \[
      \Rightarrow g \in R[x]
    \]
    primitiv $:= \gcd(a_i)=1$ 
\end{enumerate}
\textit{Beweis}
\begin{enumerate}
  \item $v_p(h)=0$, $v_p(f) \leq 0$ und $v_p(g) \leq 0$. Daraus folgt mit Lemma
    6.80 wegen $v_p(h)=v_p(f)+v_p(g)$, dass $v_p(f)=v_p(g)=0 \Rightarrow f,g \in
    R[x]$.
  \item $\forall p\in P~~v_p(h)\geq 0, v_p(f)=0$ da $f$ primitiv. Wieder folgt
    mit Lemma 6.80 $v_p(g)\geq 0 \Rightarrow g\in R[x]$
  \end{enumerate}
  \hfill $\square$
\paragraph{Theorem 6.82} \textit{(Satz von Gauß)}
Sei $R$ faktorieller Ring
\begin{enumerate}
\item Ein Polynom $q\in R[x]$ der Form
  \begin{enumerate}[label=\alph*)]
  \item $q$ ist Primelement in $R$.
  \item $q$ ist primitiv in $R[x]$ und Primelement in $Q(R)[x]$
  \end{enumerate}
  ist Primelement in R[x].
\item Alle Primelemente in $R[x]$ sind der Form 1a oder 1b.
\item $R[x]$ ist faktoriell.
\end{enumerate}
\textit{Beweis}
\dots

\section*{Körper}
\subsection{Die Charakteristik eines Körpers}

\paragraph{Definition 7.1}
Ein kommutativer Ring $K$ heißt \textit{Körper} wenn jedes Element $x\in
K\backslash\{0\}=K^*$ invertierbar ist.

\paragraph{Beispiel 7.2}
Sei $K$ ein Körper dann gibt es einen eindeutigen Ringhomomorphismus
\[
  \psi:\Z \to K,~n\mapsto n\cdot 1
\]
nach Homomorphiesatz können wir dieses $\psi$ faktorisieren mit
\[
  \Z \os{\psi}{\to}K
\]
\[
  \Z \os{\pi}{\to}\Z/ker(\psi)
\]
\[
  \Z/ker(\psi)\os{\q{\psi}}{\to}K
\]
$\q{\psi}$ ist Monomorphismus und
da $K$ ein Körper ist folgt das $K$ ein Integritätsbereich ist.
\[
  \q{\psi}:\Z/ker(\psi)\to K
\]
kann man als Unterring von $K$ betrachten. Daraus folgt $\Z/ker(\psi)$ ist
Integritätsbereich. Da $(n)=ker(\psi)\subset\Z$ folgt 1. $(*)$, $(n)$ ist ein
Primideal
$\Rightarrow n$ ist Primzahl. 2. $(**), $$n=0 \Rightarrow ker(\psi)=(0)$.

\paragraph{Definition 7.3}
\begin{enumerate}{}
\item Wenn $(*)$ gilt, dann sagt man, dass ($char(K)=p$) $K$ ein Körper von
Charakteristik $p$ ist.
\item Wenn $(**)$ gilt, dann $\dots$~$char(K)=0$ 
\end{enumerate}

\paragraph{Bemerkung 7.4}
\[
  char(K)=p \Rightarrow \us{p-mal}{1+1+\dots+1}=0
\]

\paragraph{Beispiel 7.5}
\begin{enumerate}{}
\item $\Z \os{\psi}{\to}\Q,~n\mapsto n$ folgt $ker(\psi)=(0) \Rightarrow char(\Q)=0$
\item $\mathbb{F}_p := \Z/(p)$ wobei $p$ Primzahl ist. $\Z\os{\psi}{\to}\Z/(p)$
  ist die kanonische Projektion und damit $ker(\psi)=(p)$ und
  $char(\mathbb{F}_2)=p$.
\end{enumerate}

\paragraph{Lemma 7.6}
\begin{enumerate}{}
\item Sei $K$ ein Körper und $I\subset K$ ein Ideal $\Rightarrow I=(0)$ oder
  $I=K$
\item Jeder Homomorphismus von Körpern ist injektiv.
\end{enumerate}
\textit{Beweis} \dots

\paragraph{Bemerkung 7.7}
$F\subset K$ also $F$ ist Unterkörper von $K$. Dann folgt
\[
  char(F)=char(K)
\]

\paragraph{Bemerkung 7.8}
Sei $K$ ein Körper
\begin{itemize}{}
\item $F_1,F_2\subset K$ Unterkörper $\Rightarrow F_1\cap F_2$ ist Unterkörper.
\item $F_i\subset K~~i\in I \Rightarrow \bigcap_{i\in I}F_i$ ist ein Unterkörper.
\end{itemize}

\paragraph{Definition 7.9}
Sei $P := \bigcap_{F\subset K}F\subset K$ Unterkörper, der \textit{Primkörper}
von $K$ heißt.

\paragraph{Lemma 7.10}
Sei $K$ ein Körper und $P\in K$ ein Primkörper.
\begin{enumerate}{}
\item $char(K)=0\Leftrightarrow P=\Q$
\item $char(K)=p\Leftrightarrow P=\mathbb{F}_p$
\end{enumerate}
\textit{Beweis} \dots

\paragraph{Korollar 7.11}
Sei $L\to K$ ein Homomorphismus von Körpern. Dann gilt
\begin{enumerate}{}
\item $char(L)=char(K)$
\item Primkörper werden identifiziert
\end{enumerate}
\textit{Beweis} \dots

\subsection{Körpererweiterungen}

\paragraph{Definition 7.12}
$K\subset L$ Unterkörper.
\begin{enumerate}{}
\item Man sagt, dass $L$ eine Körpererweiterung von $K$ ist. (Bezeichnung: $L/K$)
\item Man kann $L$ als $K$-Vektorraum betrachten. $\dim_KL$ wird Grad von der
  Körpererweiterung genannt und mit $[L:K]$ bezeichnet.
\item $K/K$ heißt endlich gdw. $[L:K]<\infty$
\end{enumerate}

\paragraph{Beispiele 7.13}
\begin{enumerate}{}
\item $\R \subset \C~~\dim_\R\C=2$
\item $\Q\subset\R$ $\dim_\Q\R = \infty$
\end{enumerate}

\paragraph{Lemma 7.14}
Es seien $F\subset K\subset L$ endliche Körper. Dann gilt
\[
  [L:F]=[L:K]\cdot[K:F]
\]
\textit{Beweis} \dots

\paragraph{Definition 7.15}
Sei $L/K$ eine Körpererweiterung
\begin{enumerate}
\item $\alpha\in L ~\text{heißt algebraisch}\iff \exists f(x)\in K[x],x\neq 0: f(\alpha)=0$
\item Ansonsten heißt $\alpha$ transzendent.
\item $L/K$ heißt algebraisch $\iff \forall \alpha\in L$ algebraisch über $K$ ist.
\end{enumerate}

\paragraph{Beispiel 7.16}
\begin{enumerate}{}
\item $\C/\R~~i\in\C$ ist $i$ algebraisch?
\item $\R/\Q~~\alpha =\sqrt{2}$ ist algebraisch über $\Q$
\item $\pi\in\R$ ist transzendent über $\Q$.
\end{enumerate}

\paragraph{Lemma 7.17}
Jede endliche Körpererweiterung $L/K$ ist algebraisch. \\
\textit{Beweis}
\dots

\paragraph{Definition 7.18}
$L/K$ Körpererweiterung. $\alpha\in L$
\begin{enumerate}{}
\item  $\psi_\alpha:K[x]\to L,~f(x)\mapsto f(\alpha) ~~~Im(\psi_\alpha)=K[\alpha]$
\item $L$ ist Integritätsbereich $\Rightarrow K[\alpha]$ ist Integritätsbereich.
\item Quotientenkörper von $K[\alpha]$ bezeichnen wir mit $K(\alpha)$.
\item $\alpha$ algebraisch$/K \Rightarrow \psi_\alpha$ ist nicht injektiv.
  \[
    Ker(\psi_\alpha)=I=(p(x))=p(x)K[x]
  \]
\item $K[\alpha]=k[x]/(p(x))$ (nach Homomorphiesatz). $K[\alpha]$
  Integritätsbereich $\Rightarrow (p(x))$ Primideal $\Rightarrow p(x)$
  irreduzibel. $p(x)$ ist eindeutig bis auf Multiplikation mit $u\in K^*$
  definiert. $\Rightarrow$ man kann $p(x)$ normiert vorraussetzen. $p(x)$
  Minimalpolynom von $\alpha$ über $K$.
\end{enumerate}

\paragraph{Beispiel 7.19}
\begin{enumerate}{}
\item $\C/\R~~i\in\C$
\end{enumerate}

\paragraph{Lemma 7.20}
$L/K$ Körpererweiterung, $\alpha\in L$ algebraisch$/K$, $p(x)\in K[x]$ das
Minimalpolynom von $\alpha$. Dann gilt
\begin{enumerate}{}
\item $K[\alpha]=K(\alpha)$
\item $[K(\alpha):K]=\deg(p(x))$
\end{enumerate}
\textit{Beweis} \dots

\paragraph{Definition 7.21}
$L/K$ Körpererweiterung.$\alpha_1,\dots,\alpha_n\in L$
\begin{enumerate}{}

\item $\psi_\alpha: K[x_1,\dots,x_n]\to L,~f(x_1,\dots,x_n)\mapsto f(\alpha,\dots,\alpha_n)$~~
  $Im(\psi_\alpha)=K[\alpha_1,\dots,\alpha_n]$ Integritätsbereich.
  $Q(K[\alpha_1,\dots,\alpha_n])=:K(\alpha_1,\dots,\alpha_n)$. $K\subset
  K[\alpha_1,\dots,\alpha_n]\subset K(\alpha_1,\dots,\alpha_n)$
\item $L/K$ heißt endlich erzeugt $\Leftrightarrow \exists
  \alpha_1,\dots,\alpha_n\in L: L=K(\alpha_1,\dots,\alpha_n)$.
\end{enumerate}

\paragraph{Lemma 7.22}
Sei $L=K(\alpha_1,\dots,\alpha_n)$, mit $\alpha_i$ algebraisch$/K \Rightarrow
L/K$ ist endlich.

\paragraph{Theorem 7.23}
$F\subset K\subset L$ Körpererweiterung
\[
  L/F ~\text{ist algebraisch} \Leftrightarrow L/K~\text{und}~ K/F~\text{sind algebraisch.}
\]

\paragraph{Korollar 7.24}
$L/K$ Körpererweiterung
\[
  L_{alg}=\{\alpha\in L ~|~ \alpha~alg./K\} \Rightarrow L_{alg}~\text{ist ein Unterkörper.}
\]

\subsection{Algebraischer Abschluss eines Körpers}

\paragraph{Lemma 7.25}
Sei $f(x)\in K[x]$ ein Polynom über $K$. Es existiert eine endliche
Körpererweiterung $L/K$, in welchem $f(x)$ eine Wurzel hat.

\paragraph{Beispiel/Bemerkung 7.26}
\begin{enumerate}{}
\item $K=\R~~f(x)=x^2+1$
\item $K=\Q~~x^2-2$
\end{enumerate}

\paragraph{Korollar 7.27}
$f_1(x),\dots,f_n(x)\in K[x]$. Es ex. eine endliche Körpererweiterung, in
welcher jedes $f_i(x)$ eine Wurzel hat.

\paragraph{Definition 7.28}
Ein Körper $K$ heißt algebraisch abgeschlossen, wenn jedes Polynom $f(x)\in
K[x], \deg(f)\geq 1$ eine Wurzel in $K$ hat.

\paragraph{Bemerkung 7.29}
Äquivalent: $K$ algebraisch abgeschlossen $\Leftrightarrow \forall$ normierte
Polynome $f(x)\in K[x]$ sich als Produkt $f(x)=(x-\alpha_1)\dots(x-\alpha_n)$.

\paragraph{Beispiel 7.28}
\begin{enumerate}{}
\item $\R$ nicht algebraisch abgeschlossen (z.B. $f(x)=x^2+1$)
\item $\C$ algebraisch abgeschlossen.
\item Kann ein Körper mit endlich vielen Elementen algebraisch abgeschlossen sein?
\end{enumerate}

\paragraph{Theorem 7.29}
Jeder Körper $K$ lässt sich in einem algebraisch abgeschlossenen Körper
einbetten. D.h. $\exists L/K, L$ algebraisch abgeschlossenen.

\paragraph{Korollar 7.30}
$\forall$ Körper $K$ existiert ein algebraisch abgeschlossener Körper $\q{K}$,
sodass $\q{K}/K$ algebraisch ist.

\paragraph{Definition 7.31}
$\q{K}$ heißt algebraischer Abschluss von $K$.

\paragraph{Bemerkung 7.32}
Alle algebraische Abschlüsse sind isomorph.

\paragraph{Beispiel 7.33}
$K=\R$, $\q{\R}=\C$

\subsection{Fortsetzung von Homomorphismen}

\paragraph{Notation 7.34}
$\sigma : K \to L$ Homomorphismus von Körpern und $a^\sigma := \sigma(a)$ und
$f^\sigma(x) := \sum a_i^\sigma x^i$ wobei $f(x) \in K[x]$.

\paragraph{Definition 7.35}
$K'/K, L'/L$ Körpererweiterungen und $\sigma : K \to L$ Homomorphismus. Man
sagt, dass $\tau : K' \to L'$ eine Fortsetzung von $\sigma$ ist, wenn gilt $\tau
|_K = \sigma$. 

\paragraph{Lemma 7.36}
$K'/K, L'/L$ Körpererweiterungen und $\sigma : K \to L$ Homomorphismus. Wenn
gilt $K' = K(\alpha)$ mit $\alpha$ algebraisch über $K$, dann ist die Menge von
Homomorphismen $\tau : K' \to L'$, die $\sigma$ fortsetzen, in einer Bijektion
mit der Menge von Nullstellen von Polynomen $p^\sigma(x)$ in $L'$, wobei $p(x)
\in K[x]$ das Minimalpolynom von $\alpha$ ist.

\paragraph{Korollar 7.37}
Die Anzahl von solchen Fortsetzungen ist nicht größer als $[K(a):K] = \deg(p(x))$

\paragraph{Definition 7.38}
$K'/K, K''/K$ Körpererweiterungen. Man sagt, dass $\sigma : K' \to K''$ ein
Homomorphismus von Körpern über $K$ ist (oder ein $K$-Homomorphismus), wenn er
eine Fortsetzung von $K \os{id}{\to} K$ ist.

\paragraph{Lemma 7.39}
Sei $L/K$ eine algebraische Körpererweiterung und $\sigma: L \to L$ ein
$K$-Homomorphismus. Dann ist $\sigma$ ein Automorphismus.

\paragraph{Theorem 7.40}
Sei $K'/K$ eine algebraische Körpererweiterung und $\sigma: K \to L$ ein
Homomorphismus mit $L$ algebraisch abgeschlossen $(L=\q{L})$. Dann ex. eine
Fortsetzung $\sigma': K' \to L$.

\paragraph{Korollar 7.41}
Seien $\q{K}'/K$ und $\q{K}''/K$ zwei algebraische Abschlüsse von $K$. Dann sind
$\q{K}', \q{K}''$ isomorph über $K$.

\subsection{Zerfällungskörper}
Sei $K$ ein Körper und $f(x) \in K[x]$

\paragraph{Definition 7.42}
Ein Körper $K \subset L$ hei\t Zerfällungskörper von $f(x)$, wenn $f(x)$ lässt
sich als Produkt von linearen Faktoren schreiben $f(x)=c(x-\alpha_1) \dots
(x-\alpha_n)$ und es gilt $L = K(\alpha_1, \dots, \alpha_n)$.

\paragraph{Beispiele 7.43}
\begin{enumerate}{}
\item $x^2 + 1 \in \R[x]$ $L=\C$ und $K=\R$ dann
  $x^2+1=(x-i)(x+i)$ und $\R(i)=\C$.
\item Ist $\Q(\sqrt[3]{2}) \subset \R$ Zerfällungskörper von $x^3-2$? nein.s
\end{enumerate}

\paragraph{Lemma 7.44}
\begin{enumerate}{}
\item Jeder algebraischer Abschluss $\q{K}$ von $K$ enthält einen eindeutigen
  Körper $L$, welcher der Zerfällungskörper von $f(x)$ ist.
\item Jede Einbettung über $K$ $\sigma:L \to \q{K}$ ist ein Automorphismus von
  $L$
\item Jede zwei Zerfällungskörper sind isomorph.
\end{enumerate}

\paragraph{Definition 20.4}
Sei $\{f_i(x)\}_{i \in I}$ eine Familie von Polynomen in $K[x]$. Der
Zerfällungskörper von $\{f_i(x)\}_{i \in I}$ ist ein Körper $K \subset L$,
sodass alle $f_i(x)$ lassen sich als Produkt von linearen Faktoren schreiben und
$L$ durch alle Nullstellen von $f_i(x)$ erzeugen.

\paragraph{Lemma 20.5}
\begin{enumerate}{}
\item Jeder algebraische Abschluss $\q{K}$ von $K$ enthält einen eindeutigen
  Körper $L$, welcher Zerfällungskörper von $\{f_i(x)\}_{i \in I}$ ist.
\item \dots
\item \dots
\end{enumerate}

\subsection{Normale Körpererweiterungen}

\paragraph{Definition 20.6}
Sei $L/K$ eine algebraische Körpererweiterung, $L \subset \q{K}$. $L/K$ heißt
normal $\iff$ $\forall \sigma: L \to \q{K}$ über $K$ ist ein Automorphismus.

\paragraph{Beispiel 20.7}
\begin{enumerate}{}
\item $K=\R \subset L = \C = \q{K} = \C$ ist $\C/\R$ normal? Ja.
\item $K=\Q \subset L = \Q(\sqrt[3]{2}) \subset \q{\Q} \subset \C$? Nein.
\end{enumerate}

\paragraph{Lemma 20.8}
$L/K$ ist normal $\Leftrightarrow$ Jedes irreduzible Polynom $f(x) \in K[x]$,
das eine Wurzel in $L$ hat, lässt sich über $L$ als Produkt von linearen
Faktoren schreiben.

\paragraph{Lemma 20.4}
$L/K$ ist normal $\Leftrightarrow L$ ist ein Zerfällungskörper einer Familie von Polynomen.

\subsection{Intermezzo: Endliche Körper}

$p$ Primzahl und $K$ Körper mit $char(K)=p$. Daraus folgt es gibt eine
eindeutige Einbettung von $\mathbb{F}_p \to K$.
\paragraph{Lemma 20.5}
Die Abbildung $\phi: K \to K,~x \mapsto x^p$ ist ein Homomorphismus über $\mathbb{F}_p$.

\paragraph{Lemma 20.6}
Sei $L$ ein Körper mit $q$ Elementen. Dann $\forall \alpha \in L: \alpha^q -
\alpha = 0$.

\end{document}

%%% Local Variables:
%%% ispell-check-comments: nil
%%% ispell-local-dictionary: "german8"
%%% End: